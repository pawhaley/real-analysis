%%%%%%%%%%%%%%%%%%%%%%%%%%%%%%%%%%%%%%%%%%%%%%%%%%%%%%%%%%%%%%%%%%%%%%%%%%%%%%%%%%%%%%%
%%%%%%%%%%%%%%%%%%%%%%%%%%%%%%%%%%%%%%%%%%%%%%%%%%%%%%%%%%%%%%%%%%%%%%%%%%%%%%%%%%%%%%%
% 
% This top part of the document is called the 'preamble'.  Modify it with caution!
%
% The real document starts below where it says 'The main document starts here'.

\documentclass[12pt]{article}

\usepackage{amssymb,amsmath,amsthm}
\usepackage[top=1in, bottom=1in, left=1.25in, right=1.25in]{geometry}
\usepackage{fancyhdr}
\usepackage{enumerate}
\usepackage{color}

% Comment the following line to use TeX's default font of Computer Modern.
\usepackage{times,txfonts}

\newtheoremstyle{homework}% name of the style to be used
  {18pt}% measure of space to leave above the theorem. E.g.: 3pt
  {12pt}% measure of space to leave below the theorem. E.g.: 3pt
  {}% name of font to use in the body of the theorem
  {}% measure of space to indent
  {\bfseries}% name of head font
  {:}% punctuation between head and body
  {2ex}% space after theorem head; " " = normal interword space
  {}% Manually specify head
\theoremstyle{homework} 

% Set up an Exercise environment and a Solution label.
\newtheorem*{exercisecore}{Exercise \@currentlabel}
\newenvironment{exercise}[1]
{\def\@currentlabel{#1}\exercisecore}
{\endexercisecore}

\newcommand\W{{\color{red}\textbf{(W) (Hand this one in to David.)}}}
\newcommand\tome{{\color{red}\textbf{(Hand this one in to David.)}}}

\newcommand{\localhead}[1]{\par\smallskip\noindent\textbf{#1}\nobreak\\}%
\newcommand\solution{\localhead{Solution:}}

%%%%%%%%%%%%%%%%%%%%%%%%%%%%%%%%%%%%%%%%%%%%%%%%%%%%%%%%%%%%%%%%%%%%%%%%
%
% Stuff for getting the name/document date/title across the header
\makeatletter
\RequirePackage{fancyhdr}
\pagestyle{fancy}
\fancyfoot[C]{\ifnum \value{page} > 1\relax\thepage\fi}
\fancyhead[L]{\ifx\@doclabel\@empty\else\@doclabel\fi}
\fancyhead[C]{\ifx\@docdate\@empty\else\@docdate\fi}
\fancyhead[R]{\ifx\@docauthor\@empty\else\@docauthor\fi}
\headheight 15pt

\def\doclabel#1{\gdef\@doclabel{#1}}
\doclabel{Use {\tt\textbackslash doclabel\{MY LABEL\}}.}
\def\docdate#1{\gdef\@docdate{#1}}
\docdate{Use {\tt\textbackslash docdate\{MY DATE\}}.}
\def\docauthor#1{\gdef\@docauthor{#1}}
\docauthor{Use {\tt\textbackslash docauthor\{MY NAME\}}.}
\makeatother

% Shortcuts for blackboard bold number sets (reals, integers, etc.)
\newcommand{\Reals}{\ensuremath{\mathbb R}}
\newcommand{\Nats}{\ensuremath{\mathbb N}}
\newcommand{\Ints}{\ensuremath{\mathbb Z}}
\newcommand{\Rats}{\ensuremath{\mathbb Q}}
\newcommand{\Cplx}{\ensuremath{\mathbb C}}
%% Some equivalents that some people may prefer.
\let\RR\Reals
\let\NN\Nats
\let\II\Ints
\let\CC\Cplx

%%%%%%%%%%%%%%%%%%%%%%%%%%%%%%%%%%%%%%%%%%%%%%%%%%%%%%%%%%%%%%%%%%%%%%%%%%%%%%%%%%%%%%%
%%%%%%%%%%%%%%%%%%%%%%%%%%%%%%%%%%%%%%%%%%%%%%%%%%%%%%%%%%%%%%%%%%%%%%%%%%%%%%%%%%%%%%%
% 
% The main document start here.

% The following commands set up the material that appears in the header.
\doclabel{Math 401: Homework 4}
\docauthor{Parker Whaley}
\docdate{Due September 19, 2016}

\begin{document}
Note that I am operating under the convention that $N,n,m,i,j$ are natural numbers unless otherwise specified.
\begin{exercise}

2.2.6\\
The limit of a sequence, if it exists is unique
\end{exercise}
\begin{proof}
Suppose to the contrary that there exists a sequence $\{a_x\}^\infty_{x=1}$ that converges to two values, $a$ and $b$ where $a\neq b$.  Without loss of generality assume $a>b$.  Define $2\epsilon=a-b>0$ and note that $\epsilon>0$.  By the definition of limit of a sequence we know that there exists a $N_a$ such that for all $n\geq N_a$, $|a_n-a|<\epsilon$.  Also note that by the definition of limit of a sequence we know that there exists a $N_b$ such that for all $n\geq N_b$, $|a_n-b|<\epsilon$.  Take $N=\max(N_a,N_b)$ note that for all $n\geq N$, $|a_n-a|<\epsilon$ and $|a_n-b|<\epsilon$.  Now we see that $|a_N-a|<\epsilon$ and $|a_N-b|<\epsilon$ so $|a_N-a|+|a_N-b|<2\epsilon$ or $|a-a_N+a_N-b|\leq |a-a_N|+|a_N-b|<2\epsilon$ via the tryangle inequality.  Thus $|a-b|<2\epsilon$, and noting that $a-b>0$ we get $a-b<2\epsilon=a-b$, a contradiction.  We are forced to conclude the negation of our supposition, that there is no sequence with two limits, or that a limit to a sequence, if it exists is unique.
\end{proof}

\begin{exercise}

2.3.1(a)\\
Let $x_n\geq 0$ for all $n\in\mathbb{N}$.  If $x_n\rightarrow0$ show $\sqrt{x_n}\rightarrow0$.
\end{exercise}
\begin{proof}
Chuse a $\epsilon>0$.  Define $\omega=\epsilon^2$.  Note that $x_n\rightarrow0$ implies that there exists a $N$ such that for all $n\geq N$, $|x_n|<\omega$.  Note that $|x_n|<\omega=\epsilon^2$ implies $\sqrt{|x_n|}<\epsilon$ wich means $|\sqrt{x_n}-0|<\epsilon$ for all $n\geq N$.  Thus by definition $\sqrt{x_n}\rightarrow0$.
\end{proof}

\begin{exercise}

2.3.1(b)\\
Let $x_n\geq 0$ for all $n\in\mathbb{N}$.  If $x_n\rightarrow x$ show $\sqrt{x_n}\rightarrow \sqrt{x}$.
\end{exercise}
\begin{proof}
Assume $x\neq0$ since we have already proven the statement true in that case, furthur note that since the sequence is bounded below by $0$, $x\geq 0$\\
Chuse a $\epsilon>0$.  Define $\omega=\epsilon \sqrt{x}>0$.  Note that $x_n\rightarrow x$ implies that there exists a $N$ such that for all $n\geq N$, $|x_n-x|<\omega$.  Note that $|x_n-x|<\omega=\epsilon \sqrt{x}$ implies $|x-x_n|<\epsilon \sqrt{x}$, $|\sqrt{x}+\sqrt{x_n}||\sqrt{x}-\sqrt{x_n}|<\epsilon \sqrt{x}$.  Noting that $|\sqrt{x}+\sqrt{x_n}|=\sqrt{x}+\sqrt{x_n}\geq \sqrt{x}$ implies $\sqrt{x}|\sqrt{x}-\sqrt{x_n}| \leq|\sqrt{x}+\sqrt{x_n}||\sqrt{x}-\sqrt{x_n}|<\epsilon \sqrt{x}$.  Thus $|\sqrt{x}-\sqrt{x_n}|<\epsilon$ for all $n\geq N$.  By definition $\sqrt{x_n}\rightarrow \sqrt{x}$.
\end{proof}

\begin{exercise}

2.3.3\\
Show that if $x_n\leq y_n\leq z_n$ for all $n\in\mathbb{N}$, and $\lim x_n=\lim z_n=l$, then $\lim y_n=l$.
\end{exercise}
For this proof I need the therum that $|a|<b\Leftrightarrow -b<a<b$ where $b>0$.
\begin{proof}
There are two cases $a\geq 0$ or $a<0$.\\
Case $a\geq 0$.  In this case $|a|=a$ and our statement becomes $0\leq a<b\Leftrightarrow -b<a<b$ wich is clearly true.\\
Case $a< 0$.  In this case $|a|=-a$ and our statement becomes $0\leq -a<b\Leftrightarrow -b<a<b$ wich,noting that $0\leq -a<b$ is equivelent to $0\geq a>-b$ , is clearly true.
\end{proof}
\begin{proof}
Suppose $x_n\leq y_n\leq z_n$ for all $n\in\mathbb{N}$, and $\lim x_n=\lim z_n=l$.\\
Chuse $\epsilon>0$.  By the definition of limit of a sequence we know that there exists a $N_1$ such that for all $n\geq N_1$, $|x_n-l|<\epsilon$.  By the definition of limit of a sequence we know that there exists a $N_2$ such that for all $n\geq N_2$, $|z_n-l|<\epsilon$.  Define $N=\max(N_1,N_2)$.  Note that for all $n\geq N$, $|x_n-l|<\epsilon$ and $|z_n-l|<\epsilon$ or by the above therum $-\epsilon< x_n-l<\epsilon$ and $-\epsilon< z_n-l<\epsilon$.  Note that $-\epsilon<x_n-l\leq y_n-l\leq z_n-l<\epsilon$ thus $-\epsilon<y_n-l<\epsilon$ and so $|y_n-l|<\epsilon$ for all $n>N$.  Thus by the definition of the limit of a sequence $\lim y_n=l$.
\end{proof}

\begin{exercise}

2.3.6\\
Find what $b_n=n-\sqrt{n^2+2n}$ converges to.
\end{exercise}
\begin{proof}
Note that $b_n=\frac{(n-\sqrt{n^2+2n})(n+\sqrt{n^2+2n})}{n+\sqrt{n^2+2n}}=\frac{-2n}{n+\sqrt{n^2+2n}}=\frac{-2}{1+\sqrt{1+2/n}}=\frac{a_n}{c_n}$, where $a_n=-2$ and $c_n=1+\sqrt{1+2/n}=d_n+e_n$, where $d_n=1$ and $e_n=\sqrt{1+2/n}=\sqrt{f_n}$, where $f_n=1+2/n$.  Noting that $1/n\rightarrow 0$ we see that $f_n\rightarrow 1$.  Using 2.3.1 we see that $e_n\rightarrow \sqrt{1}=1$.  By inspection $d_n\rightarrow 1$ and so by the algebreic limit therum $c_n\rightarrow 2$.  Noting that $a_n\rightarrow -2$ and that $c_n\not\rightarrow 0$ we see by the algebreic limit therum $b_n\rightarrow \frac{-2}{2}=-1$.
\end{proof}

\begin{exercise}

2.3.9(a)\\
Let $(a_n)$ be a bounded sequence, and assume $\lim b_n=0$.  Show that $\lim (a_nb_n)=0$.  Why are we not allowed to use the algebreic limit therum to do this?
\end{exercise}
Firstly this is outside of the algebreic limit therum entiarly since we are not garenteed that $a_n$ has a limit.

\begin{proof}
Suppose $(a_n)$ to be a bounded sequence, and $\lim b_n=0$.\\
Chuse $\epsilon>0$.  Since $(a_n)$ is bounded there exists a $M>0$ such that $|a_n|\leq M$ for all $n$.  By the definition of limit there exists a $N$ such that for all $n\geq N$, $|b_n|<\epsilon/M$, since $\epsilon/M>0$.  Note that $|a_nb_n-0|=|a_nb_n|=|a_n||b_n|\leq M|b_n|<\epsilon$ for all $n\geq N$, thus by the definition of limit $\lim (a_nb_n)=0$.
\end{proof}

\begin{exercise}

2.3.10(a)\\
If $\lim(a_n-b_n)=0$ then $\lim a_n=\lim b_n$.
\end{exercise}
Couterexample, consider the case $a_n=b_n=n$.  In this case $\lim(a_n-b_n)=\lim (0)=0$.  However $\lim a_n=\lim n$ wich does not exist and so the statement $\lim a_n=\lim b_n$ cannot be true.

\begin{exercise}

2.3.10(b)\\
If $\lim b_n=b$ then $\lim |b_n|=|b|$.
\end{exercise}
\begin{proof}
Chuse $\epsilon>0$.  By the definition of limit there exists a $N$ such that for all $n\geq N$, $|b_n-b|<\epsilon$.  Recall that we proved on the first homework that  $||a| - |b|| \leq |a - b|$, thus $||b_n|-|b||\leq |b_n-b|<\epsilon$ for all $n\geq N$.  By the definiton of limit $\lim |b_n|=|b|$.
\end{proof}

\begin{exercise}

2.3.10(c)\\
If $\lim a_n=a$ and $\lim(b_n-a_n)=0$ then $\lim b_n=a$.
\end{exercise}
\begin{proof}
Define $s_n=b_n-a_n$.  Note that $s_n\rightarrow 0$ and $a_n\rightarrow a$.  By the algebreic limit therum $b_n=(s_n+a_n)\rightarrow a+0=a$ thus $b_n\rightarrow a$.
\end{proof}

\begin{exercise}

2.3.10(d)\\
If $a_n\rightarrow 0$ and $|b_n-b|\leq a_n$ for all $n$ then $b_n\rightarrow b$.
\end{exercise}
\begin{proof}
Suppose $a_n\rightarrow 0$ and $|b_n-b|\leq a_n$ for all $n$.  Note that $0\leq |b_n-b|\leq a_n$ thus $|a_n|=a_n$.  Choose $\epsilon>0$.  By the definition of limit there exists a $N$ such that for all $n\geq N$, $|a_n|<\epsilon$.  Note that $|b_n-b|\leq a_n=|a_n|<\epsilon$ for all $n\geq N$.  Thus by the definition of limit $b_n\rightarrow b$.
\end{proof}

\begin{exercise}

2.4.1(a)\\
Prove that the sequence $x_1=3$ and
$$x_{n+1}=\frac{1}{4-x_n}
$$
converges.
\end{exercise}
\begin{proof}
I will use the monotone convergence therum.  So what I need to show is that our sequence is bounded and that our sequence is monotonic.\\\\
Suppose $0\leq x_n \leq 3$.  Note that $-0\geq -x_n \geq -3$, $4\geq 4-x_n \geq 1$, and since $4-x_n\geq 1>0$ we see $0\leq 1/4\leq1/( 4-x_n) \leq 1\leq 3$.  Thus $0\leq x_{n+1} \leq 3$.  Noting that $0\leq x_1=3\leq 3$, we conclude by induction that all $x_n$ are in $0\leq x_n \leq 3$.  Thus $|x_n|\leq 3$ for all n and so the sequence is bounded.\\\\
I will prove the sequence is monotonic decreasing by induction.\\
In the base case is $x_n\geq x_{n+1}$?  Well that would be, for $n=1$, $3\geq \frac{1}{4-3}=1$.  So it is monotonic decreasing in the base case.\\
Suppose $x_n\geq x_{n+1}$.  Note that $x_n\geq x_{n+1}$, $-x_n\leq -x_{n+1}$, $4-x_n\leq 4-x_{n+1}$, and noting that $4-x_n\geq1>0$ since the sequence is bounded by 3, $\frac{1}{4-x_n}=x_{n+1}\geq \frac{1}{4-x_{n+1}}=x_{n+2}$.  So I have shown that if $x_n\geq x_{n+1}$ we can conclude that $x_{n+1}\geq x_{n+2}$, and so by induction I conclude that $x_n\geq x_{n+1}$ for all $n$ and thus the sequence is monotonic decreasing.\\\\
By the monotone convergence therum we can conclude that the sequence converges.
\end{proof}
\begin{exercise}

2.4.1(b)\\
Given the sequence $x_n\rightarrow l$.  Prove that the sequence $s_n=x_{n+1}\rightarrow l$
\end{exercise}
\begin{proof}
Choose $\epsilon>0$.  By the definition of limit there exists a $N$ such that for all $m\geq N$, $|x_m-l|<\epsilon$.  Choose $n\geq N$ let $m=n+1\geq N$.  Note that $|x_m-l|<\epsilon$, $|x_{n+1}-l|<\epsilon$, $|s_n-l|<\epsilon$.  Thus by the definiton of limit $s_n\rightarrow l$.
\end{proof}
\begin{exercise}

2.4.1(c)\\
Given the sequence $x_n\rightarrow l$ and $s_n=x_{n+1}\rightarrow l$, where $x_1=3$ and
$$x_{n+1}=\frac{1}{4-x_n}
$$
find $l$.
\end{exercise}












\newpage

Note that I am operating under the convention that $N,n,m,i,j$ are natural numbers unless otherwise specified.
\begin{exercise}

2.3.5 \W\\
Let $(x_n)$ and $(y_n)$ be given, and define $(z_n)$ to be the sequence $(x_1,y_1,x_2...)$.  Prove that $(z_n)$ is convergent if and only if $\lim x_n=\lim y_n$.
\end{exercise}

\begin{proof}
Note that we can formalize this definition as
$$z_n=\begin{cases}
x_{(n+1)/2} & n\in$ odds$\\
y_{n/2}&$otherwise$\\
\end{cases}$$
We are asked in this proof to prove a "if and only if" statement, basically prove a double implication.  I will break this up into proving two implications, first $\lim x_n=\lim y_n$ implies $(z_n)$ is convergent, and second $(z_n)$ is convergent implies $\lim x_n=\lim y_n$.\\\\
Suppose $\lim x_n=\lim y_n=l$.\\
Chuse $\epsilon>0$.  By the definition of limit of a sequence we know that there exists a $N_1$ such that for all $n\geq N_1$, $|x_n-l|<\epsilon$.  By the definition of limit of a sequence we know that there exists a $N_2$ such that for all $n\geq N_2$, $|y_n-l|<\epsilon$.  Define $N=2*(\max(N_1,N_2))$.  Chuse $n\geq N$.  There are two possibilities, eather $n\in$ odd or $n\notin$ odd.\\
Case $n\in$ odd.  In this case $z_n=x_{(n+1)/2}$.  Note that $(n+1)/2>n/2\geq N/2\geq N_1$ and so $|z_n-l|=|x_{(n+1)/2}-l|<\epsilon$.\\
Case $n\notin$ odd.  In this case $z_n=y_{n/2}$.  Note that $n/2\geq N/2\geq N_2$ and so $|z_n-l|=|y_{n/2}-l|<\epsilon$.\\
So $|z_n-l|<\epsilon$ for all $n\geq N$ and thus $z_n$ will converge.\\\\
Suppose $(z_n)$ is convergent.\\
let $l=\lim z_n$.\\
Chuse $\epsilon>0$.  By the definition of limit of a sequence we know that there exists a $N$ such that for all $m\geq N$, $|z_m-l|<\epsilon$.  Chuse a $n\geq N$.  let $m=2n-1$.  Note that $m\geq n\geq N$ and that $m\in$ odds, so $z_m=x_{(m+1)/2}=x_n$.  Since $m\geq N$, $|x_n-l|=|z_m-l|<\epsilon$.  Thus $x_n$ converges to $l$.\\
Chuse $\epsilon>0$.  By the definition of limit of a sequence we know that there exists a $N$ such that for all $m\geq N$, $|z_m-l|<\epsilon$.  Chuse a $n\geq N$.  let $m=2n$.  Note that $m> n\geq N$ and that $m\notin$ odds, so $z_m=y_{m/2}=y_n$.  Since $m\geq N$, $|y_n-l|=|z_m-l|<\epsilon$.  Thus $y_n$ converges to $l$.\\
We conclude that $\lim x_n=\lim y_n$.\\\\
We can now conclude $(z_n)$ is convergent if and only if $\lim x_n=\lim y_n$.

\end{proof}
\end{document}