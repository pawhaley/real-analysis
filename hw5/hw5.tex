%%%%%%%%%%%%%%%%%%%%%%%%%%%%%%%%%%%%%%%%%%%%%%%%%%%%%%%%%%%%%%%%%%%%%%%%%%%%%%%%%%%%%%%
%%%%%%%%%%%%%%%%%%%%%%%%%%%%%%%%%%%%%%%%%%%%%%%%%%%%%%%%%%%%%%%%%%%%%%%%%%%%%%%%%%%%%%%
% 
% This top part of the document is called the 'preamble'.  Modify it with caution!
%
% The real document starts below where it says 'The main document starts here'.

\documentclass[12pt]{article}

\usepackage{amssymb,amsmath,amsthm}
\usepackage[top=1in, bottom=1in, left=1.25in, right=1.25in]{geometry}
\usepackage{fancyhdr}
\usepackage{enumerate}
\usepackage{color}

% Comment the following line to use TeX's default font of Computer Modern.
\usepackage{times,txfonts}

\newtheoremstyle{homework}% name of the style to be used
  {18pt}% measure of space to leave above the theorem. E.g.: 3pt
  {12pt}% measure of space to leave below the theorem. E.g.: 3pt
  {}% name of font to use in the body of the theorem
  {}% measure of space to indent
  {\bfseries}% name of head font
  {:}% punctuation between head and body
  {2ex}% space after theorem head; " " = normal interword space
  {}% Manually specify head
\theoremstyle{homework} 

% Set up an Exercise environment and a Solution label.
\newtheorem*{exercisecore}{Exercise \@currentlabel}
\newenvironment{exercise}[1]
{\def\@currentlabel{#1}\exercisecore}
{\endexercisecore}

\newcommand\W{{\color{red}\textbf{(W) (Hand this one in to David.)}}}
\newcommand\tome{{\color{red}\textbf{(Hand this one in to David.)}}}

\newcommand{\localhead}[1]{\par\smallskip\noindent\textbf{#1}\nobreak\\}%
\newcommand\solution{\localhead{Solution:}}

%%%%%%%%%%%%%%%%%%%%%%%%%%%%%%%%%%%%%%%%%%%%%%%%%%%%%%%%%%%%%%%%%%%%%%%%
%
% Stuff for getting the name/document date/title across the header
\makeatletter
\RequirePackage{fancyhdr}
\pagestyle{fancy}
\fancyfoot[C]{\ifnum \value{page} > 1\relax\thepage\fi}
\fancyhead[L]{\ifx\@doclabel\@empty\else\@doclabel\fi}
\fancyhead[C]{\ifx\@docdate\@empty\else\@docdate\fi}
\fancyhead[R]{\ifx\@docauthor\@empty\else\@docauthor\fi}
\headheight 15pt

\def\doclabel#1{\gdef\@doclabel{#1}}
\doclabel{Use {\tt\textbackslash doclabel\{MY LABEL\}}.}
\def\docdate#1{\gdef\@docdate{#1}}
\docdate{Use {\tt\textbackslash docdate\{MY DATE\}}.}
\def\docauthor#1{\gdef\@docauthor{#1}}
\docauthor{Use {\tt\textbackslash docauthor\{MY NAME\}}.}
\makeatother

% Shortcuts for blackboard bold number sets (reals, integers, etc.)
\newcommand{\Reals}{\ensuremath{\mathbb R}}
\newcommand{\Nats}{\ensuremath{\mathbb N}}
\newcommand{\Ints}{\ensuremath{\mathbb Z}}
\newcommand{\Rats}{\ensuremath{\mathbb Q}}
\newcommand{\Cplx}{\ensuremath{\mathbb C}}
%% Some equivalents that some people may prefer.
\let\RR\Reals
\let\NN\Nats
\let\II\Ints
\let\CC\Cplx

%%%%%%%%%%%%%%%%%%%%%%%%%%%%%%%%%%%%%%%%%%%%%%%%%%%%%%%%%%%%%%%%%%%%%%%%%%%%%%%%%%%%%%%
%%%%%%%%%%%%%%%%%%%%%%%%%%%%%%%%%%%%%%%%%%%%%%%%%%%%%%%%%%%%%%%%%%%%%%%%%%%%%%%%%%%%%%%
% 
% The main document start here.

% The following commands set up the material that appears in the header.
\doclabel{Math 401: Homework 5}
\docauthor{Parker Whaley}
\docdate{Due October 5, 2016}

\begin{document}
Note that I am operating under the convention that $N,n,m,i,j$ are natural numbers unless otherwise specified.
\begin{exercise}

Suppose $\{n_j\}_{j=1}^\infty$ is a sequence of natural numbers such that $n_j < n_{j+1}$ for all $j \in \mathbb{N}$.  Show that $n_j \geq j$ for all $j \in \mathbb{N}$.
\end{exercise}
\begin{proof}
Suppose $\{n_j\}_{j=1}^\infty$ is a sequence of natural numbers such that $n_j < n_{j+1}$ for all $j \in \mathbb{N}$.  I will prove that $n_j \geq j$ for all $j \in \mathbb{N}$ by induction on $j$.\\
Base case $j=1$.  Note that $n_1$ is a natural number, and by construction the smallest natural number is $1$.  Thus in the base case $n_j \geq j$.\\
Suppose $n_j \geq j$.  Note that $n_{j+1}>n_j$, thus $n_{j+1}-n_j>0$,noting that the integers are closed under subtraction I conclude that $n_{j+1}-n_j\in\mathbb{Z}$, thus $n_{j+1}-n_j\geq 1$ and $n_{j+1}\geq 1+n_j\geq 1+j$.
\end{proof}

\begin{exercise}

Show that a subsequence of a convergent sequence converges to the same limit. Be sure
to use the previous problem in your proof!
\end{exercise}
\begin{proof}
Suppose $\{a_n\} \rightarrow l$ and it has a subsequence $b_j$.  By the definition of subsequence we can express $b_j=a_{n_j}$ where $\{n_j\}$ is a sequence of natural numbers such that $n_j < n_{j+1}$.  Choose $\epsilon>0$.  Since $\{a_n\} \rightarrow l$ there must exist a $N\in\mathbb{N}$ such that for any $n\geq N$, $|a_n-l|<\epsilon$.  Note that for any $j\geq N$, $n_j\geq j$ thus $n_j\geq N$ and so $|b_{j}-l|=|a_{n_j}-l|<\epsilon$.  Thus $|b_{j}-l|<\epsilon$ for all $j\geq N$ by the definition of convergence $b_j\rightarrow l$.
\end{proof}

\begin{exercise}

2.4.4\\
Prove NIP using MCT.
\end{exercise}
\begin{proof}
Axium: if a sequence is monotone and bounded it converges.\\
Suppose we have sets $I_n=[a_n,b_n]$ defined for all $n\in\mathbb{N}$, where $a_n\leq a_{n+1}\leq b_{n+1}\leq b_n$.\\
Note that for $n=1$, $a_1\leq a_{n}\leq b_{n}\leq b_1$.  Suppose $a_1\leq a_{n}\leq b_{n}\leq b_1$.  Note that $a_1\leq a_{n}\leq a_{n+1}\leq b_{n+1}\leq  b_{n}\leq b_1$ and thus $a_1\leq a_{n+1}\leq b_{n+1}\leq b_1$.  By induction we are forced to conclude that $a_1\leq a_{n}\leq b_{n}\leq b_1$ for all $n\in \mathbb{N}$.\\
Noting that $a_n$ is bounded and monotonic we can conclude that it converges to some values $a_n\rightarrow a$.\\
Suppose there existed a $c\in \mathbb{N}$ where $a\not\in I_c$.  There are two sensibilities eater $a<a_c$ or $b_c<a$, I will proceed to prove both of these are impossible.\\
Suppose $a<a_c$.  Choose $\epsilon=a_c-a>0$.  By the definition of limit there exists a $N\in\mathbb{N}$ such that for all $n\geq N$, $|a_n-a|<\epsilon$.  Note that $N>c$ since if $N\leq c$ we could conclude $|a_c-a|<\epsilon$ and $|a_c-a|=|\epsilon|=\epsilon$ a contradiction.  Note that $a_N\geq a_c> a$ since $a_n$ is monotonic increasing, thus $\epsilon>|a_N-a|=a_N-a\geq a_c-a=\epsilon$.  A contradiction thus it is impossible that $a<a_c$, and we conclude that $b_c<a$.\\
Choose $\epsilon=a-b_c>0$.  By the definition of limit there exists a $N\in\mathbb{N}$ such that for all $n\geq N$, $|a_n-a|<\epsilon$.  Note that $N>c$ since if $N\leq c$ we could conclude $|a_c-a|<\epsilon$ and $a_c\leq b_c<a$, $|a_c-a|=a-a_c\geq a-b_c=\epsilon$ a contradiction.  Note that $a_N\leq b_N\leq b_c < a$ since $b_n$ is monotonic decreasing, thus $\epsilon>|a_N-a|=a-a_N\geq a-b_c=\epsilon$ a contradiction.  We are thus forced to conclude that for all $c\in\mathbb{N}$ $a\in I_c$.  Thus $a\in\cap I_n$, proving that there exists at least one value in $\cap I_n$, the nested interval property.
\end{proof}

\begin{exercise}

2.4.5(a)\\
Let $x_1=2$ 
$$ x_{n+1} =\frac{1}{2} \biggr(x_n+\frac{2}{x_n}\biggr)$$
Show $(x_n)^2\geq 2$ prove $x_n-x_{n+1}\geq 0$.  Conclude $x_n\rightarrow \sqrt{2}$.
\end{exercise}
\begin{proof}
Note that for $n=1$, $(x_n)^2=4\geq 2$.  Suppose $(x_n)^2\geq 2$.  Note that
$$ x_{n+1} =\frac{1}{2} \biggr(x_n+\frac{2}{x_n}\biggr)$$
$$ (x_{n+1})^2 =\frac{1}{4} \biggr((x_n)^2+2\cdot x_n\cdot \frac{2}{x_n}+\frac{4}{(x_n)^2}\biggr)$$
$$= \frac{(x_n)^2}{4}+1+\frac{1}{(x_n)^2}$$
$$= 1+\frac{(x_n)^4+4}{4(x_n)^2}$$
Note that there exists some $\epsilon\geq 0$ where $2+\epsilon=(x_n)^2$.
$$= 1+\frac{(2+\epsilon)^2+4}{4(2+\epsilon)}$$
$$= 1+\frac{\epsilon^2+4\epsilon+8}{4(2+\epsilon)}$$
$$\geq 1+\frac{4(2+\epsilon)}{4(2+\epsilon)}$$
$$\geq 2$$
Thus by induction $(x_n)^2\geq 2$.\\\\
Note 
$$(x_{n})^2\geq 2$$
$$x_{n}\geq \frac{2}{x_{n}}$$
$$\frac{1}{x_{n}}\leq \frac{x_{n}}{2}$$
$$0\leq \frac{x_{n}}{2}-\frac{1}{x_{n}}$$
$$0\leq x_{n}-\frac{x_{n}}{2}-\frac{1}{x_{n}}$$
$$0\leq x_{n}-\frac{1}{2}\biggr(x_{n}+\frac{2}{x_{n}}\biggr)$$
$$0\leq x_{n}-x_{n+1}$$\\\\
Since $x_n$ is monotonic decreasing and bounded below we know it converges to some value, lets call it $l$.  Note that $x_n\rightarrow l$ and $x_{n+1}\rightarrow l$ thus $x_nx_{n+1}\rightarrow l^2$ or in other words $(x_n^2+2)/2\rightarrow l^2$.  Thus $x_n^2\rightarrow 2l^2-2$ however $x_n^2\rightarrow l^2$ so $l^2=2l^2-2$ or $l^2=2$.  Therefore this sequence converges on $\sqrt{2}$.
\end{proof}

\begin{exercise}

2.4.5(b)\\
Modify the original sequence so it converges to $\sqrt{c}$. ($c>0$)
\end{exercise}
Let $x_1=\max(c,1)$ 
$$ x_{n+1} =\frac{1}{2} \biggr(x_n+\frac{c}{x_n}\biggr)$$
\begin{proof}
Note that for $n=1$, $(x_n)^2=\max(c^2,1)\geq c$.  Suppose $(x_n)^2\geq c$.  Note that
$$ x_{n+1} =\frac{1}{2} \biggr(x_n+\frac{c}{x_n}\biggr)$$
$$ (x_{n+1})^2 =\frac{1}{4} \biggr((x_n)^2+2\cdot x_n\cdot \frac{c}{x_n}+\frac{c^2}{(x_n)^2}\biggr)$$
$$= \frac{1}{4} \biggr((x_n)^2+2c+\frac{c^2}{(x_n)^2}\biggr)$$
$$= \frac{1}{4} \biggr(2c+(x_n)^2+\frac{c^2}{(x_n)^2}\biggr)$$
Note that there exists some $\epsilon\geq 0$ where $c+\epsilon=(x_n)^2$.
$$= \frac{1}{4} \biggr(2c+\frac{(c+\epsilon)^2+c^2}{c+\epsilon}\biggr)$$
$$= \frac{1}{4} \biggr(2c+\frac{c^2+2c\epsilon+\epsilon^2+c^2}{c+\epsilon}\biggr)$$
$$\geq \frac{1}{4} \biggr(2c+\frac{2c^2+2c\epsilon}{c+\epsilon}\biggr)$$
$$\geq \frac{1}{4} (2c+2c)=c$$
Thus by induction $(x_n)^2\geq c$.\\\\
Note 
$$(x_{n})^2\geq c$$
$$x_{n}\geq \frac{c}{x_{n}}$$
$$\frac{1}{x_{n}}\leq \frac{x_{n}}{c}$$
$$0\leq \frac{x_{n}}{c}-\frac{1}{x_{n}}$$
$$0\leq \frac{2x_{n}}{c}-\frac{x_{n}}{c}-\frac{1}{x_{n}}$$
$$0\leq \frac{2x_{n}}{c}-\frac{2}{2c}\biggr(x_{n}+\frac{c}{x_{n}}\biggr)$$
$$0\leq \frac{2}{c}\biggr(x_{n}-\frac{1}{2}\biggr(x_{n}+\frac{c}{x_{n}}\biggr)\biggr)$$
$$0\leq x_{n}-\frac{1}{2}\biggr(x_{n}+\frac{c}{x_{n}}\biggr)$$
$$0\leq x_{n}-x_{n+1}$$\\\\
Since $x_n$ is monotonic decreasing and bounded below we know it converges to some value, lets call it $l$.  Note that $x_n\rightarrow l$ and $x_{n+1}\rightarrow l$ thus $x_nx_{n+1}\rightarrow l^2$ or in other words $(x_n^2+c)/2\rightarrow l^2$.  Thus $x_n^2\rightarrow 2l^2-c$ however $x_n^2\rightarrow l^2$ so $l^2=2l^2-c$ or $l^2=c$.  Therefore this sequence converges on $\sqrt{c}$.
\end{proof}

\begin{exercise}

2.5.6\\
Show that $a_n=b^{1/n}\rightarrow l_b$ if $b\geq 0$ and find $l_b$.
\end{exercise}
\begin{proof}
I will consider two cases $b\leq 1$ or $b>1$.\\
In the case that $b\leq 1$.\\
Note that $b\leq b^{1/2}\leq b^{1/3}\leq \dots\leq 1$.  Note that $a_n$ is monotone increasing and bounded above therefore it converges.\\
In the case that $b > 1$.\\
Note that $b>b^{1/2}>b^{1/3}>\dots>1$.  Note that $a_n$ is monotone decreasing and bounded below therefore it converges.\\
We conclude that this sequence will converge to some limit for any value $b\geq 0$.  Let's consider this for a particular $b$.  Note that $b^{1/n}=a_n\rightarrow l$ and $\sqrt{a_n}=b^{1/2n}=a_{2n}\rightarrow l$ thus $\sqrt{a_n} \rightarrow l$ and $\sqrt{a_n} \rightarrow \sqrt{l}$ so $\sqrt{l}=l$ and so $l=1$ or $l=0$.  We can see above that if $b > 1$, then $a_n>1$ and so our limit can not be $0$ thus in this case $l_b=1$.  If $b\leq 1$ and $b\neq 0$ we can see that we have a monotone increasing sequence starting above $0$ thus $0<a_1<a_n$ and we can not converge on $0$ since we never get closer then $a_1$ and so $l_b=1$.  If $b=0$ we get $a_n=0$ and thus our sequence converges to $0$.\\
In summery $a_n\rightarrow 1$ if $b\neq 0$ and $a_n\rightarrow 0$ if $b= 0$
\end{proof}
\begin{exercise}

Suppose $|a_n|\rightarrow 0$.  Show $a_n \rightarrow 0$.
\end{exercise}
Suppose $|a_n|\rightarrow 0$.\\
Choose $\epsilon>0$.  By the definition of limit there exists some $N$ such that for all $n\geq N$, $||a_n|-0|<\epsilon$.  Note that $||a_n|-0|=||a_n||=|a_n|=|a_n-0|$.  Choose $n\geq N$.  Note that $|a_n-0|=||a_n|-0|<\epsilon$.  By the definition of limit $a_n \rightarrow 0$.
\begin{exercise}

2.5.7\\
\end{exercise}
\begin{proof}
We know that $b^n\rightarrow 0$ for $0\leq b<1$.  Note that $-(b^n)\rightarrow 0$ for $0\leq b<1$.  Noting that $|b^n|=|b|^n$ for all $b$ we see that $-|b|^n\leq b^n \leq |b|^n$.  Thus if $-1<b<1$ we know that $0\leq |b|<1$ and so $-|b|^n\rightarrow 0$ and $|b|^n\rightarrow 0$.  By the squeeze therm $b^n\rightarrow 0$ if $-1<b<1$.  Suppose $b\not\in (-1,1)$.  Further suppose $b^n\rightarrow 0$.  Note that $|b^n|\rightarrow 0$ thus $|b|^n\rightarrow 0$.  Note that $1\leq |b|\leq |b|^n$.  We have a contradiction all terms in the sequence are grater than $1$ however they converge to $0$.  We are forced to conclude the negation of our supposition that $b^n\rightarrow 0$ only if $-1< b<1$.
\end{proof}

\begin{exercise}

2.6.2\\
\end{exercise}
\begin{enumerate}[(a)]
\item
$a_n=(-1)^n/n$\\
Note that this sequence is not monotone.  Also we proved in class that this sequence converges and thus by the Cauchy criterion it is Cauchy.
\item
This is impossible.  Any sequence with a unbounded sub sequence is unbounded and thus cannot converge, since all convergent sequences are bounded, and thus is not Cauchy.
\item
Suppose $a_n$ is a monotone sequence and $a_{n_j}$ is a Cauchy sub sequence of $a_n$.  Choose $\epsilon>0$ there exists $N$ such that $i,j\geq N$, $|a_{n_j}-a_{n_i}|<\epsilon$.  Choose $n,m\geq n_N\geq N$.  Define $J=\max(n,m)$.  Note that $n_J\geq \max(n,m)\geq n_N\geq N$ so $J\geq N$.  Note $\epsilon>|a_{n_J}-a_{n_N}|\geq |a_n-a_m|$ since $a_n$ is monotone.  Thus $a_n$ is Cauchy and so $a_n$ is convergent.  There are no monotone divergent sequences with a Cauchy sub sequence.
\item
$$a_n=\begin{cases}
n & n\in $ odds$\\
0 &$otherwise$
\end{cases}$$
Clearly $a_n$ is unbounded but the sub sequence $a_{2n}$ is a sequence of zeros and thus clearly converges and thus by the Cauchy criterion it is Cauchy.
\end{enumerate}
\begin{exercise}

2.6.5\\
\end{exercise}
\begin{enumerate}[(i)]
\item
Definitely not.  Consider $a_i=1/i$ and $s_n=\sum_{i=1}^n a_i$.  We have previously proven this sequence is unbounded.  Choose $\epsilon>0$.  There exists $N$ such that $1/N<\epsilon$.  Choose $n\geq N$.  $|s_{n+1}-s_n|=|a_{n+1}|=1/(n+1)<1/N<\epsilon$.  This sequence is pseudo-Cauchy and unbounded.
\item
Suppose $a_n$ and $b_n$ are pseudo-Cauchy.  Define $c_n=a_n+b_n$.\\
Choose $\epsilon>0$.  There exists $N_a$ such that for all $n>N_a$, $|a_{n+1}-a_n|<\epsilon/2$.  There exists $N_b$ such that for all $n>N_b$, $|b_{n+1}-b_n|<\epsilon/$.  Choose $n\geq \max(N_a,N_b)$.  Note that $|a_{n+1}-a_n|<\epsilon/2$ and $|b_{n+1}-b_n|<\epsilon/$.  Note that $|c_{n+1}-c_n|=|a_{n+1}-a_n+b_{n+1}-b_n|\leq|a_{n+1}-a_n|+|b_{n+1}-b_n|<\epsilon$.  Thus $c_n$ is pseudo-Cauchy.
\end{enumerate}




\newpage
\begin{exercise}

2.5.5\W\\
Assume $(a_n)$ is a bounded sequence with the property that every convergent subsequence converges to $a$.  Show that $a_n$ must converge to $a$.
\end{exercise}
\begin{proof}
Suppose that $a_n\not\rightarrow a$.  That must mean that the statement $(\forall \epsilon>0)$, $(\exists N\in\mathbb{N})$, $(\forall n\geq N)$, $|a_n-a|<\epsilon$ must be false, thus its negation is true.  We now know $(\exists \epsilon>0)$, $(\forall N\in\mathbb{N})$, $(\exists n\geq N)$, $|a_n-a|\geq \epsilon$, let's name one of the $\epsilon$ with this property call it $\epsilon_o$ so we know that $(\forall N\in\mathbb{N})$, $(\exists n\geq N)$, $|a_n-a|\geq \epsilon_o$.\\\\
author's note: the symbolic representation is necessary here to illustrate to the reader what I am doing.  putting it all in paragraph form would needlessly complicate this proof for the reader.\\\\
To summarize I now have in hand $\epsilon_o$ a positive number with the property that for any natural number $N$ there is a $n\geq N$ with the property that $|a_n-a|\geq \epsilon_o$.\\
Let's construct a subsequence, and call it $a_{n_j}$.  How I will construct this sequence $n_1$ is the first $n\geq 1$ where $|a_n-a|\geq \epsilon_o$, this must exist since $1$ is a natural number.  I define $n_{k+1}$ to be the first $n\geq n_k+1$ with the property $|a_n-a|\geq \epsilon_o$, this must exist since $n_k+1$ is a natural number.  Thus for all $j$, $|a_{n_j}-a|\geq \epsilon_o$.\\
Since $a_n$ is bounded we know that any subsequence of it will be bounded, thus $a_{n_j}$ is bounded.  By the Bolzano-Weierstrass theorem we conclude $a_{n_j}$ has a convergent subsequence lets call it $a_{n_{j_i}}$.  Note that $a_{n_{j_i}}$ is a subsequence of $a_n$.  Since $a_{n_{j_i}}$ is a convergent subsequence of $a_n$ we know $a_{n_{j_i}}\rightarrow a$.\\
Let $\epsilon=\epsilon_o>0$.  By the definition of convergence there exists a $N\in\mathbb{N}$ such that for all $i\geq N$, $|a_{n_{j_i}}-a|<\epsilon$.  Thus $|a_{n_{j_N}}-a|<\epsilon$.  Define $J=j_N$.  Note that $|a_{n_{J}}-a|<\epsilon$.  Recall from above that by construction $|a_{n_j}-a|\geq \epsilon_o$, for all $j$, including $J$!  Thus we have arrived at a contradiction, $\epsilon>|a_{n_J}-a|\geq \epsilon_o$ or $\epsilon_o>\epsilon_o$.  We conclude the negation of our supposition $a_n\rightarrow a$.
\end{proof}
\end{document}