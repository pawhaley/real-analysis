%%%%%%%%%%%%%%%%%%%%%%%%%%%%%%%%%%%%%%%%%%%%%%%%%%%%%%%%%%%%%%%%%%%%%%%%%%%%%%%%%%%%%%%
%%%%%%%%%%%%%%%%%%%%%%%%%%%%%%%%%%%%%%%%%%%%%%%%%%%%%%%%%%%%%%%%%%%%%%%%%%%%%%%%%%%%%%%
% 
% This top part of the document is called the 'preamble'.  Modify it with caution!
%
% The real document starts below where it says 'The main document starts here'.

\documentclass[12pt]{article}

\usepackage{amssymb,amsmath,amsthm}
\usepackage[top=1in, bottom=1in, left=1.25in, right=1.25in]{geometry}
\usepackage{fancyhdr}
\usepackage{enumerate}
\usepackage{color}

% Comment the following line to use TeX's default font of Computer Modern.
\usepackage{times,txfonts}

\newtheoremstyle{homework}% name of the style to be used
  {18pt}% measure of space to leave above the theorem. E.g.: 3pt
  {12pt}% measure of space to leave below the theorem. E.g.: 3pt
  {}% name of font to use in the body of the theorem
  {}% measure of space to indent
  {\bfseries}% name of head font
  {:}% punctuation between head and body
  {2ex}% space after theorem head; " " = normal interword space
  {}% Manually specify head
\theoremstyle{homework} 

% Set up an Exercise environment and a Solution label.
\newtheorem*{exercisecore}{Exercise \@currentlabel}
\newenvironment{exercise}[1]
{\def\@currentlabel{#1}\exercisecore}
{\endexercisecore}

\newcommand\W{{\color{red}\textbf{(W) (Hand this one in to David.)}}}
\newcommand\tome{{\color{red}\textbf{(Hand this one in to David.)}}}

\newcommand{\localhead}[1]{\par\smallskip\noindent\textbf{#1}\nobreak\\}%
\newcommand\solution{\localhead{Solution:}}

%%%%%%%%%%%%%%%%%%%%%%%%%%%%%%%%%%%%%%%%%%%%%%%%%%%%%%%%%%%%%%%%%%%%%%%%
%
% Stuff for getting the name/document date/title across the header
\makeatletter
\RequirePackage{fancyhdr}
\pagestyle{fancy}
\fancyfoot[C]{\ifnum \value{page} > 1\relax\thepage\fi}
\fancyhead[L]{\ifx\@doclabel\@empty\else\@doclabel\fi}
\fancyhead[C]{\ifx\@docdate\@empty\else\@docdate\fi}
\fancyhead[R]{\ifx\@docauthor\@empty\else\@docauthor\fi}
\headheight 15pt

\def\doclabel#1{\gdef\@doclabel{#1}}
\doclabel{Use {\tt\textbackslash doclabel\{MY LABEL\}}.}
\def\docdate#1{\gdef\@docdate{#1}}
\docdate{Use {\tt\textbackslash docdate\{MY DATE\}}.}
\def\docauthor#1{\gdef\@docauthor{#1}}
\docauthor{Use {\tt\textbackslash docauthor\{MY NAME\}}.}
\makeatother

% Shortcuts for blackboard bold number sets (reals, integers, etc.)
\newcommand{\Reals}{\ensuremath{\mathbb R}}
\newcommand{\Nats}{\ensuremath{\mathbb N}}
\newcommand{\Ints}{\ensuremath{\mathbb Z}}
\newcommand{\Rats}{\ensuremath{\mathbb Q}}
\newcommand{\Cplx}{\ensuremath{\mathbb C}}
%% Some equivalents that some people may prefer.
\let\RR\Reals
\let\NN\Nats
\let\II\Ints
\let\CC\Cplx

%%%%%%%%%%%%%%%%%%%%%%%%%%%%%%%%%%%%%%%%%%%%%%%%%%%%%%%%%%%%%%%%%%%%%%%%%%%%%%%%%%%%%%%
%%%%%%%%%%%%%%%%%%%%%%%%%%%%%%%%%%%%%%%%%%%%%%%%%%%%%%%%%%%%%%%%%%%%%%%%%%%%%%%%%%%%%%%
% 
% The main document start here.

% The following commands set up the material that appears in the header.
\doclabel{Math 401: Homework 3}
\docauthor{Your name here.}
\docdate{Due September 19, 2016}

\begin{document}

\begin{exercise}{1.4.7} Finish the proof of Theorem 1.4.5 by
showing that the assumption $\alpha^2>2$ contradicts the
assumption that $\alpha=\sup A$.
\end{exercise}
\begin{proof}
\end{proof}


\begin{exercise}{Supplemental 1}
Suppose for each $k\in\Nats$ that $A_k$ is at most countable.
Use the fact that $\Nats\times\Nats$ is countably infinite to
show that $\cup_{k=1}^\infty A_k$ is at most countable.  Hint:
take advantage of surjections.
\end{exercise}
\begin{proof}
First let me define a new set $B$ where $B=\{X\in A;X\neq \emptyset\}$.  Note that $\cup B=\cup_{k=1}^\infty A_k$.  Lets next deal with the case that $B=\emptyset$ in this case $\cup B=\emptyset$ and so is at most countable infinate.  Next lets consider the case that B has a finite number of elements, we proved this case in class, a union of a finite number of at most countably infinate sets is at most countably infinate.  Now we know that we are dealing with $B$ a infinate set of at most countably infinate non-empty sets.  Now I will introduce the notation $B_{k,l}$ where $B_{k,l}$ is the $l$ element of $B_k$.  Consider the function $f:\Nats\times\Nats\rightarrow B_{k,l}$ where
$$f(a,b)=\begin{cases}
B_{a,b} & B_a$ has a $b$th element$\\
B_{1,1} & $otherwise$
\end{cases}$$
Note that this function is surjective, since given a $B_{j,k}$ we see that $(j,k)$ maps to it.  There is also a surjection between each of our $B_{j,k}$ and $\cup B$ simply map the element $B_{j,k}$ to itself in $\cup B$.  From knowing that $\Nats$ and $\Nats\times \Nats$ have the same cardinality, I conclude that there is a surjection between $\Nats$ and $\Nats\times \Nats$.  Thus I can surjectively map $\Nats \rightarrow \Nats\times \Nats \rightarrow B_{j,k} \rightarrow \cup B$.  Thus $\cup B$ is at most countably infinate.
\end{proof}

\begin{exercise}{Supplemental 2}\W

Suppose $B$ is finite and $A\subseteq B$.  Show that
$A$ is empty or finite.
\end{exercise}
Consider the case where $A\neq \emptyset$.  There must exist a bijective function mapping $f:S_m\rightarrow B$, the definition of finite.  Since $A\subseteq B$ there must be a subset of $S_m$, lets call it $S_m|_A$ that has the property $f(S_m|_A)=A$.  Lets now consider the function $g:S_m|_A\rightarrow A$ where g(x)=f(x).  Note that by construction $g$ is onto, since $g(S_m|_A)=f(S_m|_A)=A$ and since $f$ is one-to-one on B we can see $g(a)=g(b)$ implies $f(a)=f(b)$ implies $a=b$, and so g is one-to-one.  Thus $g$ is bijective.  Since $S_m|_A\in \Nats$ it will have a least element.  Construct a map $h:S_m|_A\rightarrow S_l$ where the minimum of $S_m|_A$ gets mapped to 1 and the next smallest gets maped to 2 and so on until the last element maps to $l$.  We can say that this procedure is possiable since at most it could take $m$ steps and $m$ is finite.  By construction this function is onto and one-to-one.  We now have a bijection between $S_m|_A$ and $S_l$, notice that we can now bijectively map $A$ to $S_m|_A$ and $S_m|_A$ to $S_l$ thus by definition $A$ is finite.


\begin{exercise}{1.5.10 (a) (c)}\noindent\par

\begin{enumerate}[(a)]
\item Let $C\subseteq[0,1]$ be uncountable.  Show that
there exists $a\in(0,1)$ such that $C\cap [a,1]$ is uncountable.
\item[(c)] Determine, with proof, if the same statement remains
true replacing uncountable with infinite.
\end{enumerate}
\end{exercise}
\begin{proof}[Proof (a)]
\end{proof}
\begin{proof}[Proof (b)]
\end{proof}

\begin{exercise}{Supplemental 3}
Show that the set of a finite subsets of $\Nats$ is countably infinite.
Hint: Let $A_k$ be the set of all subsets of $\Nats$ with
no more than $k$ elements.  Show that each $A_k$ is countably infinite.
\end{exercise}
\begin{proof}
\end{proof}

\begin{exercise}{2.2.2} Verify using the definition of convergence
the following limits.
\begin{enumerate}[(a)]
\item $\displaystyle \lim_{n\rightarrow\infty} \frac{2n+1}{5n+4} = \frac{2}{5}$.
\item $\displaystyle \lim_{n\rightarrow\infty} \frac{2n^2}{n^3+3} = 0$.
\item $\displaystyle \lim_{n\rightarrow\infty} \frac{\sin(n^2)}{\surd{3}{n}}=0$.
\end{enumerate}
\end{exercise}
\begin{proof}[Proof (a)]
\end{proof}
\begin{proof}[Proof (b)]
\end{proof}
\begin{proof}[Proof (c)]
\end{proof}

\begin{exercise}{Supplemental 4}\W
Carefully prove that the sequence $(x_n)$ given by $x_n=(-1)^n$ does not converge.
\end{exercise}
\begin{proof}
\end{proof}


\end{document}