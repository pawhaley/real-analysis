%%%%%%%%%%%%%%%%%%%%%%%%%%%%%%%%%%%%%%%%%%%%%%%%%%%%%%%%%%%%%%%%%%%%%%%%%%%%%%%%%%%%%%%
%%%%%%%%%%%%%%%%%%%%%%%%%%%%%%%%%%%%%%%%%%%%%%%%%%%%%%%%%%%%%%%%%%%%%%%%%%%%%%%%%%%%%%%
% 
% This top part of the document is called the 'preamble'.  Modify it with caution!
%
% The real document starts below where it says 'The main document starts here'.

\documentclass[12pt]{article}

\usepackage{amssymb,amsmath,amsthm}
\usepackage[top=1in, bottom=1in, left=1.25in, right=1.25in]{geometry}
\usepackage{fancyhdr}
\usepackage{enumerate}
\usepackage{color}

% Comment the following line to use TeX's default font of Computer Modern.
\usepackage{times,txfonts}

\newtheoremstyle{homework}% name of the style to be used
  {18pt}% measure of space to leave above the theorem. E.g.: 3pt
  {12pt}% measure of space to leave below the theorem. E.g.: 3pt
  {}% name of font to use in the body of the theorem
  {}% measure of space to indent
  {\bfseries}% name of head font
  {:}% punctuation between head and body
  {2ex}% space after theorem head; " " = normal interword space
  {}% Manually specify head
\theoremstyle{homework} 

% Set up an Exercise environment and a Solution label.
\newtheorem*{exercisecore}{Exercise \@currentlabel}
\newenvironment{exercise}[1]
{\def\@currentlabel{#1}\exercisecore}
{\endexercisecore}

\newcommand\W{{\color{red}\textbf{(W) (Hand this one in to David.)}}}
\newcommand\tome{{\color{red}\textbf{(Hand this one in to David.)}}}

\newcommand{\localhead}[1]{\par\smallskip\noindent\textbf{#1}\nobreak\\}%
\newcommand\solution{\localhead{Solution:}}

%%%%%%%%%%%%%%%%%%%%%%%%%%%%%%%%%%%%%%%%%%%%%%%%%%%%%%%%%%%%%%%%%%%%%%%%
%
% Stuff for getting the name/document date/title across the header
\makeatletter
\RequirePackage{fancyhdr}
\pagestyle{fancy}
\fancyfoot[C]{\ifnum \value{page} > 1\relax\thepage\fi}
\fancyhead[L]{\ifx\@doclabel\@empty\else\@doclabel\fi}
\fancyhead[C]{\ifx\@docdate\@empty\else\@docdate\fi}
\fancyhead[R]{\ifx\@docauthor\@empty\else\@docauthor\fi}
\headheight 15pt

\def\doclabel#1{\gdef\@doclabel{#1}}
\doclabel{Use {\tt\textbackslash doclabel\{MY LABEL\}}.}
\def\docdate#1{\gdef\@docdate{#1}}
\docdate{Use {\tt\textbackslash docdate\{MY DATE\}}.}
\def\docauthor#1{\gdef\@docauthor{#1}}
\docauthor{Use {\tt\textbackslash docauthor\{MY NAME\}}.}
\makeatother

% Shortcuts for blackboard bold number sets (reals, integers, etc.)
\newcommand{\Reals}{\ensuremath{\mathbb R}}
\newcommand{\Nats}{\ensuremath{\mathbb N}}
\newcommand{\Ints}{\ensuremath{\mathbb Z}}
\newcommand{\Rats}{\ensuremath{\mathbb Q}}
\newcommand{\Cplx}{\ensuremath{\mathbb C}}
%% Some equivalents that some people may prefer.
\let\RR\Reals
\let\NN\Nats
\let\II\Ints
\let\CC\Cplx

%%%%%%%%%%%%%%%%%%%%%%%%%%%%%%%%%%%%%%%%%%%%%%%%%%%%%%%%%%%%%%%%%%%%%%%%%%%%%%%%%%%%%%%
%%%%%%%%%%%%%%%%%%%%%%%%%%%%%%%%%%%%%%%%%%%%%%%%%%%%%%%%%%%%%%%%%%%%%%%%%%%%%%%%%%%%%%%
% 
% The main document start here.

% The following commands set up the material that appears in the header.
\doclabel{Math 401: Homework 3}
\docauthor{Parker Whaley}
\docdate{Due September 19, 2016}

\begin{document}

\begin{exercise}{1.4.7} Finish the proof of Theorem 1.4.5 by
showing that the assumption $\alpha^2>2$ contradicts the
assumption that $\alpha=\sup A$.
\end{exercise}
\begin{proof}
Suppose for the purpose of contradiction that $\alpha^2>2$.  Note that $2\alpha\in \mathbb{R}^+$ and $\alpha^2-2\in \mathbb{R}^+$ thus $\frac{\alpha^2-2}{2\alpha}\in \mathbb{R}^+$.  There exists a natural number $n$ such that $\frac{1}{n}<\frac{\alpha^2-2}{2\alpha}$.
$$\frac{1}{n}<\frac{\alpha^2-2}{2\alpha}$$
$$\frac{2\alpha n-1}{n^2}<\frac{2\alpha n}{n^2}<\alpha^2-2$$
Noting that $2\alpha n-1>0$ and $n^2>0$ we conclude $\frac{2\alpha n-1}{n^2}>0$.
$$0<\frac{2\alpha n-1}{n^2}<\alpha^2-2$$
$$0<\frac{2\alpha}{n}-\frac{1}{n^2}<\alpha^2-2$$
$$0>-\frac{2\alpha}{n}+\frac{1}{n^2}>-\alpha^2+2$$
$$\alpha^2>\alpha^2-\frac{2\alpha}{n}+\frac{1}{n^2}>2$$
$$\alpha^2>(\alpha-1/n)^2>2$$
Note that $(\alpha-1/n)^2>2>t^2$ where $t\in A$ thus $\alpha-1/n>t$ and so $\alpha-1/n$ is a upper bound on $A$.  Also note that $\alpha>\alpha-1/n$ a contradiction we have found a upper bound on A less than $\sup A$.
\end{proof}


\begin{exercise}{Supplemental 1}
Suppose for each $k\in\Nats$ that $A_k$ is at most countable.
Use the fact that $\Nats\times\Nats$ is countably infinite to
show that $\cup_{k=1}^\infty A_k$ is at most countable.  Hint:
take advantage of surjections.
\end{exercise}
\begin{proof}
First let me define a new set $B$ where $B=\{X\in A;X\neq \emptyset\}$.  Note that $\cup B=\cup_{k=1}^\infty A_k$.  Lets next deal with the case that $B=\emptyset$ in this case $\cup B=\emptyset$ and so is at most countable infinite.  Next lets consider the case that B has a finite number of elements, we proved this case in class, a union of a finite number of at most countably infinite sets is at most countably infinite.  Now we know that we are dealing with $B$ a infinite set of at most countably infinite non-empty sets.  Now I will introduce the notation $B_{k,l}$ where $B_{k,l}$ is the $l$ element of $B_k$.  Consider the function $f:\Nats\times\Nats\rightarrow B_{k,l}$ where
$$f(a,b)=\begin{cases}
B_{a,b} & B_a$ has a $b$th element$\\
B_{1,1} & $otherwise$
\end{cases}$$
Note that this function is surjective, since given a $B_{j,k}$ we see that $(j,k)$ maps to it.  There is also a surjection between each of our $B_{j,k}$ and $\cup B$ simply map the element $B_{j,k}$ to itself in $\cup B$.  From knowing that $\Nats$ and $\Nats\times \Nats$ have the same cardinality, I conclude that there is a surjection between $\Nats$ and $\Nats\times \Nats$.  Thus I can surjectively map $\Nats \rightarrow \Nats\times \Nats \rightarrow B_{j,k} \rightarrow \cup B$.  Thus $\cup B$ is at most countably infinite.
\end{proof}

\begin{exercise}{Supplemental 2}\W

Suppose $B$ is finite and $A\subseteq B$.  Show that
$A$ is empty or finite.
\end{exercise}
Consider the case where $A\neq \emptyset$.  There must exist a bijective function mapping $f:S_m\rightarrow B$, the definition of finite.  Since $A\subseteq B$ there must be a subset of $S_m$, lets call it $S_m|_A$ that has the property $f(S_m|_A)=A$.  Lets now consider the function $g:S_m|_A\rightarrow A$ where g(x)=f(x).  Note that by construction $g$ is onto, since $g(S_m|_A)=f(S_m|_A)=A$ and since $f$ is one-to-one on B we can see $g(a)=g(b)$ implies $f(a)=f(b)$ implies $a=b$, and so g is one-to-one.  Thus $g$ is bijective.  Since $S_m|_A\in \Nats$ it will have a least element.  Construct a map $h:S_m|_A\rightarrow S_l$ where the minimum of $S_m|_A$ gets mapped to 1 and the next smallest gets mapped to 2 and so on until the last element maps to $l$.  We can say that this procedure is possible since at most it could take $m$ steps and $m$ is finite.  By construction this function is onto and one-to-one.  We now have a bijection between $S_m|_A$ and $S_l$, notice that we can now bijectively map $A$ to $S_m|_A$ and $S_m|_A$ to $S_l$ thus by definition $A$ is finite, or empty.


\begin{exercise}{1.5.10 (a) (c)}\noindent\par

\begin{enumerate}[(a)]
\item Let $C\subseteq[0,1]$ be uncountable.  Show that
there exists $a\in(0,1)$ such that $C\cap [a,1]$ is uncountable.
\item[(c)] Determine, with proof, if the same statement remains
true replacing uncountable with infinite.
\end{enumerate}
\end{exercise}
\begin{proof}[Proof (a)]
We will proceed with a proof by contradiction.  Suppose all sets $C\cap [a,1]$ where $a\in (0,1)$ is at most countably infinite.  This implies $C\cap [1/n,1]$ where $n\in \mathbb{N}$ is at most countably infinite, since $1/n\in (0,1)$.  Consider a new set $A=\cup_n [C\cap [1/n,1]]$.  As previously shown a union of at most countably infinite sets is at most countably infinite thus $A$ is at most countably infinite.  Consider an arbitrary element $b\in C-\{0\}$.  Note that there exists $i\in\mathbb{N}$ such that $1/i\leq b$ since $b\in\mathbb{R}^+$.  Thus $b\in C\cap [1/i,1]$ and so $b\in \cup_n [C\cap [1/n,1]]=A$.  Since $b$ was chosen arbitrarily from $C-\{0\}$ and shown to be in $A$ we can conclude $C-\{0\}\subseteq A$.  Noting that $A+\{0\}$ is at most countably infinite we conclude $C$ is at most countably infinite, a contradiction.  Thus there exists a $a\in (0,1)$ such that $C\cap [a,1]$ is uncountable.
\end{proof}
\begin{proof}[Proof (b)]
authors note:  By inspection it is definitely not true if we replace uncountable with infinite, but initial it is hard to see where the first proof breaks down.  However examine the statement "a union of at most countably infinite sets is at most countably infinite".  To transition this proof to the infinites instead of the uncountables we would need "a union of at most finite sets is at most finite", witch is definitely not true. For example $\cup_n \{n\}$ is a union of at most finite sets and is definitely non-finite.\\\\
It is not true that if $C\subseteq [0,1]$ is infinite that there exists a $a\in (0,1)$ where $C\cap [a,1]$ is infinite, I will prove this with a counterexample.\\
Consider the set $C=\{1/n:n\in\mathbb{N}\}$.  Note that $C\subseteq [0,1]$ and that $C$ is infinite, since it clearly has the same number of elements as $\mathbb{N}$.  Suppose for the purpose of contradiction that there existed a $a\in (0,1)$ where $C\cap [a,1]$ is infinite.  Note that $1/a\in (1,\infty)$ thus the minimum natural number grater than $a$, guaranteed to exist since there is a natural number bigger than a given real and the naturals are well ordered, is not 1 and so the number one less than the minimum natural is a natural.  In other words there exists a $i\in \mathbb{N}$ such that $i\leq 1/a< i+1$ or $1/i \geq a> 1/(i+1)$.  Now we see that $C\cap [a,1]$ is nothing other than $C-\{1/n:n>i,n\in \mathbb{N}\}$ or 
$\{1/n:n\in[1,i],n\in \mathbb{N}\}=\{1/n:n\in S_{i}\}$.  This clearly has a bijective map with $S_i$ and is therefore finite, a contradiction, thus there is no $a$ that makes $C\cap [a,1]$ infinite.
\end{proof}

\begin{exercise}{Supplemental 3}
Show that the set of a finite subsets of $\Nats$ is countably infinite.
Hint: Let $A_k$ be the set of all subsets of $\Nats$ with
no more than $k$ elements.  Show that each $A_k$ is countably infinite.
\end{exercise}
\begin{proof}
Define $A_k$ to be the set of all subsets of $\Nats$ with
no more than $k$ elements.\\
I will proceed with induction on k.\\
In the base case $k=1$.  Note that $A_k=\{\{n\}:n\in \Nats\}$ in this case $A_k$ is countably infinite since there is clearly a bijective map to $\Nats$, $\{n\}\rightarrow n$.\\
Suppose $A_k$ is countably infinite.  Define $A_{kn}$ to be the $n$th element of $A_k$, we can order $A_k$ since there exists a bijective map to $\Nats$.  Define $B_{ni}=A_{kn}+\{i\}$, note that $B_{ni}$ has at most $k+1$ elements since, by construction of $A_k$, $A_{kn}$ has at most $k$ elements.  Note that $\{B_{ni}:n\in\Nats, i\in\Nats\}$ has a surjective map to $\Nats\times \Nats$ and thus a surjective map exists to $\Nats$ and so is at most countably infinite also note that $A_{k+1}=\{B_{ni}:n\in\Nats, i\in\Nats\}$, since all non-empty sets made up of $k+1$ or fewer elements are simply a non-empty set containing $k$ or fewer elements adding in a new element (note that the new element could already exist in the set), thus $A_{k+1}$ is at most countably infinite.  Since $A_k\subseteq A_{k+1}$ we know $A_{k+1}$ is not finite, thus $A_{k+1}$ is countably infinite.\\
By induction all sets $A_k$ are countably infinite
\end{proof}

\begin{exercise}{2.2.2} Verify using the definition of convergence
the following limits.
\begin{enumerate}[(a)]
\item $\displaystyle \lim_{n\rightarrow\infty} \frac{2n+1}{5n+4} = \frac{2}{5}$.
\item $\displaystyle \lim_{n\rightarrow\infty} \frac{2n^2}{n^3+3} = 0$.
\item $\displaystyle \lim_{n\rightarrow\infty} \frac{\sin(n^2)}{\sqrt[3]{n}}=0$.
\end{enumerate}
\end{exercise}
\begin{proof}[Proof (a)]
Chose an $\epsilon >0$.  Select $N\in\mathbb{N}$ such that $(\frac{3}{5\epsilon}-4)/(5)<N$ this can be done since given a real number there is a natural bigger than it.  Note that:
$$(\frac{3}{5\epsilon}-4)/(5)<N$$
$$\frac{3}{5\epsilon}-4<5N$$
$$\frac{3}{5\epsilon}<5N+4$$
$$\epsilon>\frac{3/5}{5N+4}$$
Chose a $n>N$.  Note that $|\frac{2n+1}{5n+4}-\frac{2}{5}|=|\frac{2n+1-\frac{2}{5}(5n+4)}{5n+4}|=|\frac{1-\frac{8}{5}}{5n+4}|=|\frac{-3/5}{5n+4}|=\frac{3/5}{5n+4}\leq \frac{3/5}{5N+4}<\epsilon$.
\end{proof}
\begin{proof}[Proof (b)]
Chose an $\epsilon >0$.  Select $N\in\mathbb{N}$ such that $1/N<\epsilon/2$ this can be done since $\epsilon/2>0$.  Chose a $n>N$.  Note that $|\frac{2n^2}{n^3+3}-0|=\frac{2n^2}{n^3+3}<\frac{2n^2}{n^3}=\frac{2}{n}<\frac{2}{N}<2\epsilon/2=\epsilon$.
\end{proof}
\begin{proof}[Proof (c)]
Chose an $\epsilon >0$.  Select $N\in\mathbb{N}$ such that $1/N<\epsilon^3$ this can be done since $\epsilon^3>0$.  Chose a $n>N$.  Note that $|\frac{\sin(n^2)}{\sqrt[3]{n}}-0|\leq \frac{1}{\sqrt[3]{n}}<\frac{1}{\sqrt[3]{N}}=(\frac{1}{N})^{1/3}<(\epsilon^3)^{1/3}=\epsilon$.
\end{proof}

\begin{exercise}{Supplemental 4}\W
Carefully prove that the sequence $(x_n)$ given by $x_n=(-1)^n$ does not converge.
\end{exercise}
\begin{proof}
Suppose that $x_n=(-1)^n$ converges to $a$.  Let's let $\epsilon=.1$.  from the definition of limit we know there exists a $N$ such that for all $n>N$ $|(-1)^n-a|<\epsilon$.  Take $n_1$ to be a odd natural number grater than $N$.  Take $n_2=n_1+1$.  Note that $|(-1)^{n_1}-a|<\epsilon$ and $|(-1)^{n_2}-a|<\epsilon$, thus $|1-a|<\epsilon$ and $|-1-a|<\epsilon$.  Consider two cases $a \geq 0$ and $a<0$.  In the case $a \geq 0$, $.1=\epsilon>|-1-a|=1+a\geq 1$ in this case we have a contradiction $.1>1$.  In the case $a <0$, $.1=\epsilon>|1-a|=1-a\geq 1$ in this case we have a contradiction $.1>1$.  Thus this series cannot converge.
\end{proof}


\end{document}