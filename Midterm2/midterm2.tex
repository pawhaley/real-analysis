%%%%%%%%%%%%%%%%%%%%%%%%%%%%%%%%%%%%%%%%%%%%%%%%%%%%%%%%%%%%%%%%%%%%%%%%%%%%%%%%%%%%%%%
%%%%%%%%%%%%%%%%%%%%%%%%%%%%%%%%%%%%%%%%%%%%%%%%%%%%%%%%%%%%%%%%%%%%%%%%%%%%%%%%%%%%%%%
% 
% This top part of the document is called the 'preamble'.  Modify it with caution!
%
% The real document starts below where it says 'The main document starts here'.

\documentclass[12pt]{article}

\usepackage{amssymb,amsmath,amsthm}
\usepackage[top=1in, bottom=1in, left=1.25in, right=1.25in]{geometry}
\usepackage{fancyhdr}
\usepackage{enumerate}
\usepackage{color}

% Comment the following line to use TeX's default font of Computer Modern.
\usepackage{times,txfonts}

\newtheoremstyle{homework}% name of the style to be used
  {18pt}% measure of space to leave above the theorem. E.g.: 3pt
  {12pt}% measure of space to leave below the theorem. E.g.: 3pt
  {}% name of font to use in the body of the theorem
  {}% measure of space to indent
  {\bfseries}% name of head font
  {:}% punctuation between head and body
  {2ex}% space after theorem head; " " = normal interword space
  {}% Manually specify head
\theoremstyle{homework} 

% Set up an Exercise environment and a Solution label.
\newtheorem*{exercisecore}{Exercise \@currentlabel}
\newenvironment{exercise}[1]
{\def\@currentlabel{#1}\exercisecore}
{\endexercisecore}

\newcommand\W{{\color{red}\textbf{(W) (Hand this one in to David.)}}}
\newcommand\tome{{\color{red}\textbf{(Hand this one in to David.)}}}

\newcommand{\localhead}[1]{\par\smallskip\noindent\textbf{#1}\nobreak\\}%
\newcommand\solution{\localhead{Solution:}}

%%%%%%%%%%%%%%%%%%%%%%%%%%%%%%%%%%%%%%%%%%%%%%%%%%%%%%%%%%%%%%%%%%%%%%%%
%
% Stuff for getting the name/document date/title across the header
\makeatletter
\RequirePackage{fancyhdr}
\pagestyle{fancy}
\fancyfoot[C]{\ifnum \value{page} > 1\relax\thepage\fi}
\fancyhead[L]{\ifx\@doclabel\@empty\else\@doclabel\fi}
\fancyhead[C]{\ifx\@docdate\@empty\else\@docdate\fi}
\fancyhead[R]{\ifx\@docauthor\@empty\else\@docauthor\fi}
\headheight 15pt

\def\doclabel#1{\gdef\@doclabel{#1}}
\doclabel{Use {\tt\textbackslash doclabel\{MY LABEL\}}.}
\def\docdate#1{\gdef\@docdate{#1}}
\docdate{Use {\tt\textbackslash docdate\{MY DATE\}}.}
\def\docauthor#1{\gdef\@docauthor{#1}}
\docauthor{Use {\tt\textbackslash docauthor\{MY NAME\}}.}
\makeatother

% Shortcuts for blackboard bold number sets (reals, integers, etc.)
\newcommand{\Reals}{\ensuremath{\mathbb R}}
\newcommand{\Nats}{\ensuremath{\mathbb N}}
\newcommand{\Ints}{\ensuremath{\mathbb Z}}
\newcommand{\Rats}{\ensuremath{\mathbb Q}}
\newcommand{\Cplx}{\ensuremath{\mathbb C}}
%% Some equivalents that some people may prefer.
\let\RR\Reals
\let\NN\Nats
\let\II\Ints
\let\CC\Cplx

%%%%%%%%%%%%%%%%%%%%%%%%%%%%%%%%%%%%%%%%%%%%%%%%%%%%%%%%%%%%%%%%%%%%%%%%%%%%%%%%%%%%%%%
%%%%%%%%%%%%%%%%%%%%%%%%%%%%%%%%%%%%%%%%%%%%%%%%%%%%%%%%%%%%%%%%%%%%%%%%%%%%%%%%%%%%%%%
% 
% The main document start here.

% The following commands set up the material that appears in the header.
\doclabel{Math 401: Midterm2}
\docauthor{Parker Whaley}
\docdate{Due November 16, 2016}

\begin{document}
Note that I am operating under the convention that $N,n,m,i,j$ are natural numbers unless otherwise specified.  I am also operating under the convention $v_a(b)=\{x\in\mathbb{R}:|x-b|<a\}$
\begin{exercise}
1
Suppose $f : A \rightarrow \mathbb{R}$ and $c$ is a limit point of $A$. Suppose $f (x) \geq 0$ for all $x \in A$ and that
$lim_{x \rightarrow c}f (x)$ exists. Show that the limit is non-negative. Provide two proofs, one $\epsilon$ - $\delta$
style, and the other using the sequential characterization of limits.
\end{exercise}
\begin{proof}
Suppose $f : A \rightarrow \mathbb{R}$ and $c$ is a limit point of $A$. Suppose $f (x) \geq 0$ for all $x \in A$ and that
$lim_{x \rightarrow c}f (x)=L$ exists.\\
Suppose $L<0$.  Define $\epsilon=-L/2>0$.  There must exist a $\delta>0$ such that for all $x\in A$ where $0<|x-c|<\delta$, $|f(x)-L|<\epsilon$.  Note that $c$ is a limit point of $A$, thus $v_\delta(c)\cap A-\{c\}\neq \emptyset$.  Take one of the elements of this set $a\in v_\delta(c)\cap A-\{c\}$.  Note that $a\neq c$.  Note that $a\in A$.  Note $|a-c|<\delta$.  Thus $|f(a)-L|<\epsilon$ and so $L-\epsilon<f(a)<L+\epsilon=L/2<0$.  A contradiction, we know that $f(a)\geq 0$ and now we have $f(a)<0$.  We now conclude $L\geq 0$.
\end{proof}
\begin{exercise} 2
Let $a_n$ be a sequence of numbers such that for some $M \in \mathbb{R}$, $\sum^\infty_{n=1} a_nM^n$ converges. Suppose
that $|x| < M$. Show that $\sum^\infty_{n=1} a_nx^n$ converges absolutely. Give an example to show
that divergence is possible if $|x| = |M|$. Hint: ($a_nM^n$) converges to zero, and is hence
bounded.
\end{exercise}
\begin{proof}
Let $a_n$ be a sequence of numbers such that for some $M \in \mathbb{R}$, $\sum^\infty_{n=1} a_nM^n$ converges. Suppose
that $|x| < M$.\\
Note that by our supposition $|x| < M$ we can conclude $0< M$.  Define $-1<0\leq r=\frac{|x|}{M}<1$.  Note that $\lim_{n\rightarrow \infty} a_nM^n=0$ since $\sum^\infty_{n=1} a_nM^n$ converges.  Note that a convergent sequence is bounded so we can define $N\in\mathbb{R}$ such that $|a_nM^n|<N$ for all $n$.  Note $|r|<1$, thus $\sum_{n=1}^\infty N r^n$ converges lets call it's limit $L$.  Noting that $0\leq N$ and $0\leq r$ we can conclude $0\leq N r^n$ for $n\in\mathbb{N}$ and thus $\sum_{n=1}^k N r^n$ is monotone increasing.  Note $\sum_{n=1}^k N r^n\leq \sum_{n=1}^\infty N r^n=L$.  Define $S_k=\sum^k_{n=1} |a_nx^n|$.  Note that $S_k\leq S_k+|a_{k+1}x^{k+1}|=S_{k+1}$ for $k\in\mathbb{N}$ and thus $S_k$ is monotone increasing.  Note that $|a_n|<\frac{N}{M^n}$, recall $|a_nM^n|<N$ and $0\leq M$.  Note that $S_k=\sum^k_{n=1} |a_nx^n|<\sum^k_{n=1} \frac{N}{M^n}|x^n|=\sum^k_{n=1} N r^n\leq L$.  By the MCT $S_k$ converges and thus $\sum^\infty_{n=1} a_nx^n$ converges absolutely.

\end{proof}
For a example where $|x| = |M|$ and $\sum^\infty_{n=1} a_nx^n$ does not converges absolutely, consider the case $a_n=(-1)^n1/n$, $M=1$, $x=1$.  Note that $\sum^\infty_{n=1} a_nM^n=\sum^\infty_{n=1} (-1)^n1/n$ converges.  However $\sum^\infty_{n=1} |a_nx^n|=\sum^\infty_{n=1} 1/n$ does not converge.
\begin{exercise} 3
Suppose $f : (0,1]=A \rightarrow\mathbb{R}$ is uniformly continuous. Show that $lim_{x\rightarrow 0} f (x)$ exists.
\end{exercise}
\begin{proof}
Note that 0 is a limit point of $(0,1]$.  Take a arbitrary sequence $a_n\in A$ where $a_n\rightarrow 0$.  Choose $\epsilon>0$.  Note that there exists $\delta>0$ such that for all $x,y\in A$ if $|x-y|<\delta$ then $|f(x)-f(y)|<\epsilon/2$
\end{proof}

















\end{document}