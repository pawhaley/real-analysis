%%%%%%%%%%%%%%%%%%%%%%%%%%%%%%%%%%%%%%%%%%%%%%%%%%%%%%%%%%%%%%%%%%%%%%%%%%%%%%%%%%%%%%%
%%%%%%%%%%%%%%%%%%%%%%%%%%%%%%%%%%%%%%%%%%%%%%%%%%%%%%%%%%%%%%%%%%%%%%%%%%%%%%%%%%%%%%%
% 
% This top part of the document is called the 'preamble'.  Modify it with caution!
%
% The real document starts below where it says 'The main document starts here'.

\documentclass[12pt]{article}

\usepackage{amssymb,amsmath,amsthm}
\usepackage[top=1in, bottom=1in, left=1.25in, right=1.25in]{geometry}
\usepackage{fancyhdr}
\usepackage{enumerate}
\usepackage{color}

% Comment the following line to use TeX's default font of Computer Modern.
\usepackage{times,txfonts}

\newtheoremstyle{homework}% name of the style to be used
  {18pt}% measure of space to leave above the theorem. E.g.: 3pt
  {12pt}% measure of space to leave below the theorem. E.g.: 3pt
  {}% name of font to use in the body of the theorem
  {}% measure of space to indent
  {\bfseries}% name of head font
  {:}% punctuation between head and body
  {2ex}% space after theorem head; " " = normal interword space
  {}% Manually specify head
\theoremstyle{homework} 

% Set up an Exercise environment and a Solution label.
\newtheorem*{exercisecore}{Exercise \@currentlabel}
\newenvironment{exercise}[1]
{\def\@currentlabel{#1}\exercisecore}
{\endexercisecore}

\newcommand\W{{\color{red}\textbf{(W) (Hand this one in to David.)}}}
\newcommand\tome{{\color{red}\textbf{(Hand this one in to David.)}}}

\newcommand{\localhead}[1]{\par\smallskip\noindent\textbf{#1}\nobreak\\}%
\newcommand\solution{\localhead{Solution:}}

%%%%%%%%%%%%%%%%%%%%%%%%%%%%%%%%%%%%%%%%%%%%%%%%%%%%%%%%%%%%%%%%%%%%%%%%
%
% Stuff for getting the name/document date/title across the header
\makeatletter
\RequirePackage{fancyhdr}
\pagestyle{fancy}
\fancyfoot[C]{\ifnum \value{page} > 1\relax\thepage\fi}
\fancyhead[L]{\ifx\@doclabel\@empty\else\@doclabel\fi}
\fancyhead[C]{\ifx\@docdate\@empty\else\@docdate\fi}
\fancyhead[R]{\ifx\@docauthor\@empty\else\@docauthor\fi}
\headheight 15pt

\def\doclabel#1{\gdef\@doclabel{#1}}
\doclabel{Use {\tt\textbackslash doclabel\{MY LABEL\}}.}
\def\docdate#1{\gdef\@docdate{#1}}
\docdate{Use {\tt\textbackslash docdate\{MY DATE\}}.}
\def\docauthor#1{\gdef\@docauthor{#1}}
\docauthor{Use {\tt\textbackslash docauthor\{MY NAME\}}.}
\makeatother

% Shortcuts for blackboard bold number sets (reals, integers, etc.)
\newcommand{\Reals}{\ensuremath{\mathbb R}}
\newcommand{\Nats}{\ensuremath{\mathbb N}}
\newcommand{\Ints}{\ensuremath{\mathbb Z}}
\newcommand{\Rats}{\ensuremath{\mathbb Q}}
\newcommand{\Cplx}{\ensuremath{\mathbb C}}
%% Some equivalents that some people may prefer.
\let\RR\Reals
\let\NN\Nats
\let\II\Ints
\let\CC\Cplx

%%%%%%%%%%%%%%%%%%%%%%%%%%%%%%%%%%%%%%%%%%%%%%%%%%%%%%%%%%%%%%%%%%%%%%%%%%%%%%%%%%%%%%%
%%%%%%%%%%%%%%%%%%%%%%%%%%%%%%%%%%%%%%%%%%%%%%%%%%%%%%%%%%%%%%%%%%%%%%%%%%%%%%%%%%%%%%%
% 
% The main document start here.

% The following commands set up the material that appears in the header.
\doclabel{Math 401: Midterm2}
\docauthor{Parker Whaley}
\docdate{Due November 16, 2016}

\begin{document}
Note that I am operating under the convention that $N,n,m,i,j$ are natural numbers unless otherwise specified.  I am also operating under the convention $v_a(b)=\{x\in\mathbb{R}:|x-b|<a\}$
\begin{exercise}
1
Suppose $f : A \rightarrow \mathbb{R}$ and $c$ is a limit point of $A$. Suppose $f (x) \geq 0$ for all $x \in A$ and that
$lim_{x \rightarrow c}f (x)$ exists. Show that the limit is non-negative. Provide two proofs, one $\epsilon$ - $\delta$
style, and the other using the sequential characterization of limits.
\end{exercise}
\begin{proof}
Suppose $f : A \rightarrow \mathbb{R}$ and $c$ is a limit point of $A$. Suppose $f (x) \geq 0$ for all $x \in A$ and that
$lim_{x \rightarrow c}f (x)=L$ exists.\\
Suppose $L<0$.  Define $\epsilon=-L/2>0$.  There must exist a $\delta>0$ such that for all $x\in A$ where $0<|x-c|<\delta$, $|f(x)-L|<\epsilon$.  Note that $c$ is a limit point of $A$, thus $v_\delta(c)\cap A-\{c\}\neq \emptyset$.  Take one of the elements of this set $a\in v_\delta(c)\cap A-\{c\}$.  Note that $a\neq c$.  Note that $a\in A$.  Note $|a-c|<\delta$.  Thus $|f(a)-L|<\epsilon$ and so $L-\epsilon<f(a)<L+\epsilon=L/2<0$.  A contradiction, we know that $f(a)\geq 0$ and now we have $f(a)<0$.  We now conclude $L\geq 0$.
\end{proof}
\begin{exercise} 2
Let $a_n$ be a sequence of numbers such that for some $M \in \mathbb{R}$, $\sum^\infty_{n=1} a_nM^n$ converges. Suppose
that $|x| < M$. Show that $\sum^\infty_{n=1} a_nx^n$ converges absolutely. Give an example to show
that divergence is possible if $|x| = |M|$. Hint: ($a_nM^n$) converges to zero, and is hence
bounded.
\end{exercise}
\begin{proof}
Let $a_n$ be a sequence of numbers such that for some $M \in \mathbb{R}$, $\sum^\infty_{n=1} a_nM^n$ converges. Suppose
that $|x| < M$.\\
Note that by our supposition $|x| < M$ we can conclude $0< M$.  Define $-1<0\leq r=\frac{|x|}{M}<1$.  Note that $\lim_{n\rightarrow \infty} a_nM^n=0$ since $\sum^\infty_{n=1} a_nM^n$ converges.  Note that a convergent sequence is bounded so we can define $N\in\mathbb{R}$ such that $|a_nM^n|<N$ for all $n$.  Note $|r|<1$, thus $\sum_{n=1}^\infty N r^n$ converges lets call it's limit $L$.  Noting that $0\leq N$ and $0\leq r$ we can conclude $0\leq N r^n$ for $n\in\mathbb{N}$ and thus $\sum_{n=1}^k N r^n$ is monotone increasing.  Note $\sum_{n=1}^k N r^n\leq \sum_{n=1}^\infty N r^n=L$.  Define $S_k=\sum^k_{n=1} |a_nx^n|$.  Note that $S_k\leq S_k+|a_{k+1}x^{k+1}|=S_{k+1}$ for $k\in\mathbb{N}$ and thus $S_k$ is monotone increasing.  Note that $|a_n|<\frac{N}{M^n}$, recall $|a_nM^n|<N$ and $0\leq M$.  Note that $S_k=\sum^k_{n=1} |a_nx^n|<\sum^k_{n=1} \frac{N}{M^n}|x^n|=\sum^k_{n=1} N r^n\leq L$.  By the MCT $S_k$ converges and thus $\sum^\infty_{n=1} a_nx^n$ converges absolutely.

\end{proof}
For a example where $|x| = |M|$ and $\sum^\infty_{n=1} a_nx^n$ does not converges absolutely, consider the case $a_n=(-1)^n1/n$, $M=1$, $x=1$.  Note that $\sum^\infty_{n=1} a_nM^n=\sum^\infty_{n=1} (-1)^n1/n$ converges.  However $\sum^\infty_{n=1} |a_nx^n|=\sum^\infty_{n=1} 1/n$ does not converge.
\begin{exercise} 3
Suppose $f : (0,1]=A \rightarrow\mathbb{R}$ is uniformly continuous. Show that $lim_{x\rightarrow 0} f (x)$ exists.
\end{exercise}
\begin{proof}
Note that 0 is a limit point of $(0,1]$.  Take a arbitrary sequence $a_n\in A$ where $a_n\rightarrow 0$.  Define $f_n=f(a_n)$.  Choose $\epsilon>0$.  Note that there exists $\delta>0$ such that for all $x,y\in A$ if $|x-y|<\delta$ then $|f(x)-f(y)|<\epsilon$.  There exists $N\in \mathbb{N}$ such that for all $m, n\geq N$, $|a_n-a_m|<\delta$, Caushey criterion for a sequence.  Choose $m, n\geq N$.  Note that $|f_n-f_m|<\epsilon$.  Thus $f_n$ is a Caushey sequence and thus converges.  We now have esablished that for any sequence $a_n\in A$ where $a_n\rightarrow 0$, $f(a_n)$ converges to some limit.\\\\
Consider two sequences $a_n\in A$ where $a_n\rightarrow 0$ and $b_n\in A$ where $b_n\rightarrow 0$.  Note that $f(a_n)$ converges and $f(b_n)$ converges, define there limits to be $L_a$ and $L_b$.  Define a new sequence $c_n$ to be the shuffeled sequence $a_1,b_1,a_2,b_2,\cdots$.  Note that by the shuffeled sequence theorem $c_n\rightarrow 0$.  We conclude $f(c_n)$ converges.  Note that $f(c_n)$ is the shuffeled sequence of $f(a_n)$ and $f(b_n)$ thus by the shuffeled sequence theorem $L_a=L_b=L$.  We selected two sequences in $A$ converging on 0 and showed that the sequences made of the functional evaluations of those sequences converge to the same value.  We can now conclude all sequences made this way converge to the same value, call this value $L$.  We have now proven that for a arbitrary sequence $a_n\in A$ where $a_n\rightarrow 0$, $f(a_n)$ converges to $L$.  By the sequential charecterization of limits we can conclude that $lim_{x\rightarrow 0} f (x)$ exists and is $L$
\end{proof}

\begin{exercise} 4
Abbott 4.3.11
\end{exercise}
Let $f$ be a function defined on all of $\mathbb{R}$, and assume there exists a $c$ where $0<c<1$ such that $|f(x)-f(y)|\leq c|x-y|$ for all $x,y\in\mathbb{R}$.
\begin{enumerate}[(a)]
\item
Show that $f$ is continuous on all $\mathbb{R}$.\\
Choose $x\in\mathbb{R}$.  Choose $\epsilon>0$.  Define $\delta=\epsilon$.  Choose $y\in\mathbb{R}$ such that $|x-y|<\delta$.  Note that $|f(x)-f(y)|\leq c|x-y|\leq |x-y|<\delta=\epsilon$.  Thus $f$ is continuous on all $\mathbb{R}$.
\item
Show that for all $y_1\in\mathbb{R}$ the sequence $y_n=f(y_{n-1})$ converges to some $y$.\\
Choose $y_1\in\mathbb{R}$.  Note that $|y_{n+1}-y_{n+2}|\leq c|y_n-y_{n+1}|$.\\\\
Define $k=\frac{1}{c}|y_2-y_1|$.  For $n=1$, $|y_{n+1} - y_n|\leq c^nk$ would mean $|y_2-y_1|\leq |y_2-y_1|$, clearly true.  Suppose $|y_{n+1} - y_n|\leq c^nk$.  Note that $|y_{n+2} - y_{n+1}| \leq c|y_{n+1} - y_{n}|$, so $|y_{n+2} - y_{n+1}|\leq c^{n+1}k$.  By induction $|y_{n+1} - y_n|\leq c^nk$ for all natural numbers $n$.\\\\
Note that for $m=1$, $|y_{n+m} - y_n|\leq c^{n-1}k\sum_{i=1}^m c^i$ would mean $|y_{n+1} - y_n|\leq c^nk$, clearly true.  Suppose for some $m$, $|y_{n+m} - y_n|\leq c^{n-1}k\sum_{i=1}^m c^i$.  Note that $|y_{n+m+1} - y_{n+m}|\leq c^{n+m}k$.  Note that $|y_{n+m+1} - y_n|=|y_{n+m+1} - y_{n+m}+y_{n+m} - y_n|\leq|y_{n+m+1} - y_{n+m}|+|y_{n+m} - y_n| \leq c^{n-1}k\sum_{i=1}^m c^i+c^{n+m}k=c^{n-1}k(\sum_{i=1}^mc^i+c^{m+1})=c^{n-1}k\sum_{i=1}^{m+1}c^i$.  By induction on $m$ I conclude $|y_{n+m} - y_n|\leq c^{n-1}k\sum_{i=1}^m c^i$ for all natural numbers $m$.\\\\
Note that $\sum_{i=1}^mc^i\leq \sum_{i=1}^{\infty}c^i=L$ (see geometric series Pg. 73).  Thus $|y_{n+m} - y_n|\leq c^{n-1}kL$ for all $m$ and $n$.\\\\
Note that $c^n$ is a monotone decreasing sequence and is bounded below by 0, thus by MCT it will converge to $l$.  Also note that $c^{n+1}$ will converge to the same value, thus $l=cl$ and so $l=0$.  Thus $c^n\rightarrow 0$.\\\\
Choose $\epsilon>0$.  Noting that $\frac{c\epsilon}{KL}>0$ we can say that there exists a natural number $N$ such that for all $n\geq N$, $|c^n|<\frac{c\epsilon}{KL}$.  Choose $m>n\geq N$ Define $N$ this way.  Define $d=m-n$, note that $d\in\mathbb{N}$.  Note that $c^{n-1}kL<\epsilon$.  Note that $|y_m - y_n|=|y_{n+d} - y_n|\leq c^{n-1}kL<\epsilon$.  We now know the sequence is Caushey and therefore it will converge.
\item
Prove that any $y$ obtained in the manner above will be a fixed point, then prove that $y$ will be unique.\\
Define $y'=f(y)$.  Suppose $y'\neq y$.  Note $\epsilon=|y'-y|>0$.  Note that there exists $N\in\mathbb{N}$ such that if $n\geq N$, $|y_n-y|<\epsilon/3$.  Note that $|y_{N+1}-y'|=|f(y_N)-f(y)|\leq c|y_N-y|<\epsilon/3$.  Note that $|y-y_{N+1}|<\epsilon/3$.  Note $\epsilon=|y-y_{N+1}+y_{N+1}-y'|\leq |y-y_{N+1}|+|y_{N+1}-y'|<2\epsilon/3$, a contradiction, negate our supposition to conclude $y=f(y)$.\\\\
Suppose that $y$ is not unique.  There must exist $y_1$ and $y_2$ with the properties of $y$ and $y_1\neq y_2$.  Note that $|f(y_1)-f(y_2)|=|y_1-y_2|>c|y_1-y_2|$.  We also know that $|f(y_1)-f(y_2)|\leq c|y_1-y_2|$, a contradiction and so we must conclude that $y$ is unique.
\item
We have proven that an arbitrary element of the reals will converge to some $y$ under repeated applications of $f$.  We have also proved the uniqueness of $y$, and can now conclude that any element of the reals will converge under repeated applications of $f$ to some real number $y$.
\end{enumerate}

\begin{exercise} 5
Suppose that $f : (0,1) \rightarrow\mathbb{R}$ is continuous and that $lim_{x\rightarrow 0} f (x) = \infty$ and $lim_{x\rightarrow 1} f (x) = \infty$. Show that $f$ obtains a minimum on (0,1).
\end{exercise}
\begin{proof}
Suppose that $f : (0,1) \rightarrow\mathbb{R}$ is continuous and that $lim_{x\rightarrow 0} f (x) = \infty$ and $lim_{x\rightarrow 1} f (x) = \infty$.  Define $c_0=f(1/3)$ and $c_1=f(2/3)$.  There exists a $\delta_0>0$ such that if $|x|<\delta_0$, $f(x)>c_0+1$.  There exists a $\delta_1>0$ such that if $|x-1|<\delta_1$, $f(x)>c_1+1$.  Suppose $\delta_0> 1/3$, we would then conclude $c_0=f(1/3)>c_0+1$, a contradiction, thus $\delta_0\leq 1/3$.  Suppose $\delta_1> 1/3$, we would then conclude $c_1=f(2/3)>c_1+1$, a contradiction, thus $\delta_1\leq 1/3$.  We can now conclude $0<\delta_0\leq 1/3<2/3\leq 1-\delta_1<1$.  We now conclude $[\delta_0,1-\delta_1]\subset (0,1)$ and $[\delta_0,1-\delta_1]\neq \emptyset$ also note that $[\delta_0,1-\delta_1]$ is compact.  By the extream value theorem there exists a $m\in [\delta_0,1-\delta_1]$ such that $f(m)\leq f(y)$ for all $y\in [\delta_0,1-\delta_1]$.  Note that $f(m)\leq c_0$ and $f(m)\leq c_1$.\\
Choose $x\in (0,1)$.  There are three posibilities, $x\in(0,\delta_0)$ or $x\in [\delta_0,1-\delta_1]$ or $x\in(1-\delta_1,1)$.\\
Suppose $x\in(0,\delta_0)$.  Note that $|x|<\delta_0$ and thus $f(m)\leq c_0<c_0+1<f(x)$.\\
Suppose $x\in[\delta_0,1-\delta_1]$.  Note that $f(m)\leq f(x)$.\\
Suppose $x\in(0,\delta_0)$.  Note that $|x|<\delta_0$ and thus $f(m)\leq c_1<c_1+1<f(x)$.\\
We conclude that $f$ accheves a minimum at $m$ in $(0,1)$.
\end{proof}
\begin{exercise} 6
Abbott 4.4.13
\end{exercise}
\begin{enumerate}[(a)]
\item
\begin{proof}
Suppose $f:A\rightarrow \mathbb{R}$ is uniformly continuous, and $\{x_n\}$ is a cauchy sequence in $A$.  Choose $\epsilon>0$.  Note that there exists $\delta>0$ such that for $x,y\in A$ if $|x-y|<\delta$ then $|f(x)-f(y)|<\epsilon$.  There exists a $N\in\mathbb{N}$ such that if $n,m\geq N$, $|x_n-x_m|<\delta$.  Choose $n,m\geq N$.  Note that $|x_n-x_m|<\delta$, thus $|f(x_n)-f(x_m)|<\epsilon$.  Conclude that $f(x_n)$ is a Cauchy sequence.
\end{proof}
\item
\begin{proof}
Let $g$ be a continuous function on the open interval $(a, b)$.\\\\
Suppose it is possible to define values $g(a)$ and $g(b)$ at the endpoints so that the extended function $g'$ is continuous on $[a, b]$.  Note that $[a, b]$ is a compact set, thus $g'$ is continuous on a compact set.  By Theorem 4.4.7 $g'$ is uniformly continuous on $[a,b]$.  Choose $\epsilon>0$.  There exists a $\delta>0$ such that for all $x,y\in [a,b]$, if $|x-y|<\delta$ then $|g'(x)-g'(y)|<\epsilon$.  Choose $x,y\in (a,b)$.  Note that $x,y\in [a,b]$ and $g'(x)=g(x)$, $g'(y)=g(y)$.  Note that if $|x-y|<\delta$ then $|g(x)-g(y)|=|g'(x)-g'(y)|<\epsilon$.  Conclude $g$ is uniformly continuous on $(a,b)$.\\\\
Note that since uniformly continuous functions preserve causheyness they will also preserve convergence of a sequence, if $x_n$ converges $f(x_n)$ will converge if $f$ is uniformly continuous.\\
Suppose that $g$ is uniformly continuous on $(a,b)$.  Select a sequence $a_n$ in $(a,b)$ where $a_n\rightarrow a$.  Define $g'(a)$ as $g(a_n)\rightarrow g'(a)$.  Select a sequence $b_n$ in $(a,b)$ where $b_n\rightarrow a$.  Define a new sequence $c_n$ to be the shuffeled sequence $a_1,b_1,a_2,b_2,\cdots$.  Note that by the shuffeled sequence theorem $c_n\rightarrow a$.  We conclude $g(c_n)$ converges.  Note that $g(c_n)$ is the shuffeled sequence of $g(a_n)$ and $g(b_n)$ thus by the shuffeled sequence theorem $g(b_n)\rightarrow g'(a)$.  By the sequential charicterization of limits $\lim_{x\rightarrow a}g(x)=g'(a)$.\\
Select a sequence $d_n$ in $(a,b)$ where $d_n\rightarrow b$.  Define $g'(b)$ as $g(d_n)\rightarrow g'(b)$.  Select a sequence $e_n$ in $(a,b)$ where $e_n\rightarrow b$.  Define a new sequence $f_n$ to be the shuffeled sequence $d_1,e_1,d_2,e_2,\cdots$.  Note that by the shuffeled sequence theorem $f_n\rightarrow b$.  We conclude $g(f_n)$ converges.  Note that $g(f_n)$ is the shuffeled sequence of $g(d_n)$ and $g(e_n)$ thus by the shuffeled sequence theorem $g(e_n)\rightarrow g'(b)$.  By the sequential charicterization of limits $\lim_{x\rightarrow b}g(x)=g'(b)$.\\
Note that the function $$g'(x)=\begin{cases} g(x) & x\neq a,b\\ g'(a) & x=a\\ g'(b) & x=b\end{cases}$$ is continuous at eavery point in $(a,b)$ since $g(x)$ is continuous on $(a,b)$ and is continuous at $a$ and $b$ since the value matches the limit at those points.\\\\
Now conclude $g$ is uniformly continuous on $(a, b)$ if and only if it is possible to define value $g(a)$ and $g(b)$ at the endpoints so that the extended function $g$ is continuous on $[a, b]$.
\end{proof}
\end{enumerate}
\begin{exercise} 7
Show that if $f : [a, b] \rightarrow \mathbb{R}$ is strictly increasing and continuous, then it has
a continuous inverse function $f^{-1}: [f(a), f(b)] \rightarrow [a, b]$. Use this result to show that $x^{1/n}$
is
continuous for each $n \in \mathbb{N}$. You may use any homework problems you have done to help
with this. But your proof must give a careful demonstration that the domain of $f^{-1}$
is the
whole interval $[f(a), f(b)]$, that the image is exactly $[a, b]$, and that $f^{-1}$
is increasing.
\end{exercise}
\begin{proof}
Suppose $f : [a, b] \rightarrow \mathbb{R}$ is strictly increasing and continuous.  Define the function $G(y)=\{x\in[a, b] :f(x)=y \}$.  Choose $y_-\in (-\infty,f(a))$.  Suppose $x\in G(y_-)$, note that $f(x)=y_-<f(a)$ and $a\leq x$ so $f(a)\leq f(x)$, a contradiction thus $G(y_-)=\emptyset$.  Choose $y_+\in (f(b),\infty)$.  Suppose $x\in G(y_+)$, note that $f(x)=y_+>f(b)$ and $x\leq b$ so $f(x)\leq f(b)$, a contradiction thus $G(y_+)=\emptyset$.  Choose $y\in [f(a),f(b)]$.  By the IVT there exists a $x$ such that $f(x)=y$, thus $G(y)\neq \emptyset$.  Suppose $x_1<x_2\in G(y)$, note $f(x_1)<f(x_2)$ and $f(x_1)=f(x_2)$, a contradiction thus $G(y)$ only has one element.\\
We have now proven that $G(y)$ has one element if $y\in[f(a), f(b)]$ and no elements if $y\not\in[f(a), f(b)]$.  Noting that $G(y)\subseteq[a, b] $ we can now define $f^{-1}:[f(a), f(b)]\rightarrow [a, b]$ defined as $f(y)$ is the element in $G(y)$.\\
Choose $x\in [a,b]$.  Note that $f(a)\leq f(x)\leq f(b)$, thus $f^{-1}(f(x))$ is defined.  Note from the definition that $x\in G(f(x))$ therfore $f^{-1}(f(x))=x$.  Choose $y\in[f(a), f(b)]$.  Note that $f^{-1}(y)$ is defined call it's value $c$. Note from the definition fo $G$ that $f(c)=y$, so $f(f^{-1}(y))=y$.  Conclude that $f^{-1}(y)$ is indeed the inverse function for $f(x)$.\\
Choose $y_1<y_2\in[f(a), f(b)]$.  Suppose $f^{-1}(y_1)\geq f^{-1}(y_2)$.  We would then conclude $f(f^{-1}(y_1))\geq f(f^{-1}(y_2))$ or $y_1\geq y_2$, a contradiction, conclude that $f^{-1}$ is strictly increasing.\\
Choose $y\in (f(a), f(b))$.  Choose $\epsilon>0$.  Define $x=f^{-1}(y)$, note $x\neq a$ and $x\neq b$.  Define $x_+=\min(b,x+\epsilon)$ and $x_-=\max(a,x-\epsilon)$.  Note that $x_-<x<x_+$.  Note that $f(x_-)<f(x)<f(x_+)$.  Define $\delta=\min(f(x)-f(x_-),f(x_+)-f(x))>0$.  Choose $y'\in [f(a), f(b)]$ such that $|y-y'|<\delta$.  Note that $y=f(x)$.  Note that $-\delta<y'-f(x)<\delta$ so $f(x_-)<y'<f(x_+)$ and $x-\epsilon\leq x_-<f^{-1}(y')<x_+\leq x+\epsilon$ or $|f^{-1}(y')-f^{-1}(y)|<\epsilon$.  So our inverse function is continuous for all $y\in (f(a), f(b))$.\\
Choose $\epsilon>0$.  Define $\delta=f(b)-f(b-\epsilon)>0$.  Choose $y\in [f(a), f(b)]$ such that $|f(b)-y|<\delta$.  Note that $y\leq f(b)$.  Note that $0 \leq f(b)-y<\delta$ or $f(b) \geq y>f(b)-\delta=f(b-\epsilon)$.  Note $b \geq f^{-1}(y)>b-\epsilon$ or $|f^{-1}(y)-f^{-1}(f(b))|<\epsilon$.  Thus $f^{-1}$ is continuous at $f(b)$.\\
Choose $\epsilon>0$.  Define $\delta=f(a+\epsilon)-f(a)>0$.  Choose $y\in [f(a), f(b)]$ such that $|f(a)-y|<\delta$.  Note that $y\geq f(a)$.  Note that $0 \leq y-f(a)<\delta$ or $f(a) \leq y<f(a)+\delta=f(a+\epsilon)$.  Note $a \leq f^{-1}(y)<a+\epsilon$ or $|f^{-1}(y)-f^{-1}(f(a))|<\epsilon$.  Thus $f^{-1}$ is continuous at $f(a)$.\\
Thus $f^{-1}$ is continuous on $[f(a),f(b)]$.\\
Consider $f:[0,M]$, $f(x)=x^n$ where $n\in\mathbb{N}$.  Note that it is a polynomial and therfore continuous.  Note that it is strictly increasing.
\end{proof}
















\end{document}