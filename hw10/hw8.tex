%%%%%%%%%%%%%%%%%%%%%%%%%%%%%%%%%%%%%%%%%%%%%%%%%%%%%%%%%%%%%%%%%%%%%%%%%%%%%%%%%%%%%%%
%%%%%%%%%%%%%%%%%%%%%%%%%%%%%%%%%%%%%%%%%%%%%%%%%%%%%%%%%%%%%%%%%%%%%%%%%%%%%%%%%%%%%%%
% 
% This top part of the document is called the 'preamble'.  Modify it with caution!
%
% The real document starts below where it says 'The main document starts here'.

\documentclass[12pt]{article}

\usepackage{amssymb,amsmath,amsthm}
\usepackage[top=1in, bottom=1in, left=1.25in, right=1.25in]{geometry}
\usepackage{fancyhdr}
\usepackage{enumerate}
\usepackage{color}

% Comment the following line to use TeX's default font of Computer Modern.
\usepackage{times,txfonts}

\newtheoremstyle{homework}% name of the style to be used
  {18pt}% measure of space to leave above the theorem. E.g.: 3pt
  {12pt}% measure of space to leave below the theorem. E.g.: 3pt
  {}% name of font to use in the body of the theorem
  {}% measure of space to indent
  {\bfseries}% name of head font
  {:}% punctuation between head and body
  {2ex}% space after theorem head; " " = normal interword space
  {}% Manually specify head
\theoremstyle{homework} 

% Set up an Exercise environment and a Solution label.
\newtheorem*{exercisecore}{Exercise \@currentlabel}
\newenvironment{exercise}[1]
{\def\@currentlabel{#1}\exercisecore}
{\endexercisecore}

\newcommand\W{{\color{red}\textbf{(W) (Hand this one in to David.)}}}
\newcommand\tome{{\color{red}\textbf{(Hand this one in to David.)}}}

\newcommand{\localhead}[1]{\par\smallskip\noindent\textbf{#1}\nobreak\\}%
\newcommand\solution{\localhead{Solution:}}

%%%%%%%%%%%%%%%%%%%%%%%%%%%%%%%%%%%%%%%%%%%%%%%%%%%%%%%%%%%%%%%%%%%%%%%%
%
% Stuff for getting the name/document date/title across the header
\makeatletter
\RequirePackage{fancyhdr}
\pagestyle{fancy}
\fancyfoot[C]{\ifnum \value{page} > 1\relax\thepage\fi}
\fancyhead[L]{\ifx\@doclabel\@empty\else\@doclabel\fi}
\fancyhead[C]{\ifx\@docdate\@empty\else\@docdate\fi}
\fancyhead[R]{\ifx\@docauthor\@empty\else\@docauthor\fi}
\headheight 15pt

\def\doclabel#1{\gdef\@doclabel{#1}}
\doclabel{Use {\tt\textbackslash doclabel\{MY LABEL\}}.}
\def\docdate#1{\gdef\@docdate{#1}}
\docdate{Use {\tt\textbackslash docdate\{MY DATE\}}.}
\def\docauthor#1{\gdef\@docauthor{#1}}
\docauthor{Use {\tt\textbackslash docauthor\{MY NAME\}}.}
\makeatother

% Shortcuts for blackboard bold number sets (reals, integers, etc.)
\newcommand{\Reals}{\ensuremath{\mathbb R}}
\newcommand{\Nats}{\ensuremath{\mathbb N}}
\newcommand{\Ints}{\ensuremath{\mathbb Z}}
\newcommand{\Rats}{\ensuremath{\mathbb Q}}
\newcommand{\Cplx}{\ensuremath{\mathbb C}}
%% Some equivalents that some people may prefer.
\let\RR\Reals
\let\NN\Nats
\let\II\Ints
\let\CC\Cplx

%%%%%%%%%%%%%%%%%%%%%%%%%%%%%%%%%%%%%%%%%%%%%%%%%%%%%%%%%%%%%%%%%%%%%%%%%%%%%%%%%%%%%%%
%%%%%%%%%%%%%%%%%%%%%%%%%%%%%%%%%%%%%%%%%%%%%%%%%%%%%%%%%%%%%%%%%%%%%%%%%%%%%%%%%%%%%%%
% 
% The main document start here.

% The following commands set up the material that appears in the header.
\doclabel{Math 401: Homework 10}
\docauthor{Parker Whaley}
\docdate{Due November 23, 2016}

\begin{document}
Note that I am operating under the convention that $N,n,m,i,j$ are natural numbers unless otherwise specified.  I am also operating under the convention $v_a(b)=\{x\in\mathbb{R}:b-a<x<b+a\}$
\begin{exercise}

Abbott 5.2.3 (a,b)
\end{exercise}
\begin{enumerate}[(a)]
\item
Find from definition the derivative of $h(x)=\frac{1}{x}$.\\
Note that $\frac{1}{x}$ is defined on $\mathbb{R}-\{0\}$.  Choose $c\in \mathbb{R}-\{0\}$.  Consider the function $g(x)=\frac{h(x)-h(c)}{x-c}$.  Note that $g(x)$ is defined on $A=\mathbb{R}-\{c\}$ and thus $c$ is a limit point of the domain $A$.  Note that $g(x)=\frac{1/x-1/c}{x-c}$.  Define $d(x)=x-c$, note that $d(x)\neq 0$ for $x\in A$.  Note that $g(x)=\frac{1/x-1/(x-d(x))}{d(x)}=\frac{x-d(x)-x}{x(x-d(x))d(x)}=\frac{-1}{x(x-d(x))}$.  Note that as $x\rightarrow c$ $d(x)\rightarrow 0$ and thus by the arithmetic limit therm $g(x)\rightarrow \frac{-1}{c^2}$.  Thus by definition $h'(c)=\frac{-1}{c^2}$.
\item
Suppose $g(c)\neq 0$.  Find $(f/g)'(c)$, assuming that $f$ and $g$ are differentiable at $c$.\\
Note that $(f/g)(x)=f(x)*1/g(x)$.  Define $h(x)=1/x$.  Note $(f/g)(x)=f(x)*h(g(x))$, everywhere that $f/g$ is defined.  Note that $(f/g)(c)$ is defined.  Note that $(f/g)'(c)=f'(c)h(g(x))+f(c)h'(g(c))g'(c)$ by the chain rule and product rule.  Note that $(f/g)'(c)=\frac{f'(c)}{g(c)}+\frac{-f(c)g'(c)}{g(c)^2}=\frac{f'(c)g(c)-f(c)g'(c)}{g(c)^2}$.

\end{enumerate}

\begin{exercise}

Abbott 5.3.1
\end{exercise}
\begin{enumerate}[(a)]
\item
Suppose $f'$ exists and is continuous on $[a,b]$.  Note that $f$ is continuous on $[a,b]$.  Noting that $[a,b]$ is compact and $f'$ is a continuous mapping $f':[a,b]\rightarrow \mathbb{R}$ we can say that $f'$ achieves a minimum and a maximum in [a,b], lets call them $a$ and $b$ respectively.  Define $M=\max(-a,b)$.  Note that for all $x\in [a,b]$, $-M\leq a\leq f'(x)\leq b\leq M$ and thus $|f'(x)|\leq M$.  Choose $x\neq y\in [a,b]$.  By the mean value theorem there exists a $c\in[a,b]$ such that $f'(c)=\frac{f(x)-f(y)}{x-y}$.  Note that $\frac{f(x)-f(y)}{x-y}=f'(c)\leq M$.  Conclude that $f$ is Lipschitz on $[a,b]$.
\item
Suppose $f'$ exists and is continuous on $[a,b]$.  Suppose that $|f'(x)|<1$ for all $x\in[a,b]$.  Note that $f'$ achieves a maximum and a minimum in [a,b], take the one with the largest absolute value, lets call it $M$ with associated value $x_M$.  Note that $|M|=|f'(x_M)|<1$, and $|f'(c)|\leq M$ for all $c\in [a,b]$.  Choose $x,y\in[a,b]$.  If $x=y$ note that $|f(x)-f(y)|=0=|M||x-y|$.  Suppose $x\neq y$.  Note that $\frac{|f(x)-f(y)|}{|x-y|}=|f'(c)|$ for some $c\in [a,b]$.  Thus $\frac{|f(x)-f(y)|}{|x-y|}\leq |M|$ and so $|f(x)-f(y)|\leq |M||x-y|$.  Thus $f$ is a contraction function.

\end{enumerate}
\begin{exercise}

Abbott 5.3.2
\end{exercise}
Suppose $f$ is differentiable on some interval $A$.  Suppose further that $f'(x)\neq 0$ for all $x\in A$.  Suppose $f(x)=f(y)$ for some $x\neq y\in A$.  Note that there exists a $c\in A$ such that $f'(c)=\frac{f(x)-f(y)}{x-y}=\frac{0}{x-y}=0$.  We have reached a contradiction and thus conclude $f(x)\neq f(y)$ for all $x\neq y\in A$, or that the function $f$ is one-to-one.\\
The converse is not true, consider $f(x)=x^3$ on $A=[-1,1]$.  Clearly this function is differentiable with derivative $f'(x)=3x^2$ and also one-to-one.  However note that $f'(0)=0$.


\begin{exercise}

Abbott 5.3.6 (a,b)
\end{exercise}
\begin{enumerate}[(a)]
\item
Let $g:A=[0,a]\rightarrow \mathbb{R}$ be differentiable, $g(0)=0$, and $|g'(x)|\leq M$ for all $x\in A$.  Choose $x\in A$.  If $x=0$ then $|g(x)|=0=Mx$.  Suppose $x\neq 0$.  Note that there exists a $c\in[0,a]$ such that $g'(c)=\frac{g(x)-g(0)}{x-0}=\frac{g(x)}{x}$.  Note that $\frac{|g(x)|}{x}=|\frac{g(x)}{x}|\leq M$ and thus $|g(x)|\leq Mx$.
\item
Let $h:A=[0,a]\rightarrow \mathbb{R}$ be twice differentiable, $h'(0)=h(0)=0$, and $|h''(x)|\leq M$ for all $x\in A$.  Define $g:A\rightarrow \mathbb{R}$ as $g(x)=h'(x)$.  Note that $g(0)=0$, and $|g'(x)|\leq M$ for all $x\in A$, thus $|g(x)|\leq Mx$.  Note $|h'(x)|\leq Mx$ for all $x\in A$.  Define $f(x)=Mx^2/2$.  Note that $f'(x)=Mx$.  Note that $|h'(x)|/|f'(x)|\leq 1$ for all $x\in (0,a]$.  Choose $x\in [0,a]$.  If $x=0$ clearly $|h(x)|\leq Mx^2/2$.  Suppose $x\neq 0$.  By the general mean value theorem there exists a $c\in (0,x)$ such that $\frac{h'(c)}{f'(c)}=\frac{h(x)-h(0)}{f(x)-f(0)}=\frac{h(x)}{f(x)}$ thus $|\frac{h(x)}{f(x)}|\leq 1$ or $|h(x)|\leq Mx^2/2$.
\end{enumerate}
\begin{exercise}

Abbott 5.3.7
\end{exercise}
\begin{proof}
Suppose $f$ is differentiable on a interval $A$ and that $f'(x)\neq 0$.  Further suppose $f$ has at least two fixed points, $a$, $b$.  Note that there exists a $c\in A$ such that $f'(c)=\frac{f(a)-f(b)}{a-b}=\frac{a-b}{a-b}=1$.  We have a contradiction and so conclude that there is at most one fixed point.
\end{proof}

\begin{exercise}

Abbott 6.2.1 (a,b)
\end{exercise}
Let $f_n(x)=\frac{nx}{1+nx^2}$.
\begin{enumerate}
\item
Find the point-wise limit.\\
Choose $x\in(0,\infty)$.  Consider the sequence $f_n(x)$.  Note that $\frac{nx}{1+nx^2}=\frac{x}{1/n+x^2}\rightarrow \frac{x}{x^2}=\frac{1}{x}$.
\item
Suppose uniform convergence on $(0,\infty)$.  There exists $N\in\mathbb{N}$ such that if $n\geq N$ then for all $x\in (0,\infty)$, $|f_n(x)-1/x|<1$.  Choose $x=\min(1/2,1/\sqrt{N})$.  Note that $|f_n(x)-1/x|<1$ so $\frac{1}{x(1+nx^2)}<\epsilon$ or $1<\epsilon x(1+nx^2)<\epsilon x(1+1)<\epsilon=1$, a contradiction thus $f$ is not uniformly convergent.
\end{enumerate}

\begin{exercise}

Abbott 6.2.7
\end{exercise}
Suppose $f$ is uniformly continuous on $\mathbb{R}$.  Define $f_n(x)=f(x-1/n)$.  Choose $\epsilon>0$.  There exists a $\delta>0$ such that for $x,y\in\mathbb{R}$ if $|x-y|<\delta$ then $|f(x)-f(y)|<\epsilon$.  Define $N\in\mathbb{N}$ such that $1/N<\delta/2$.  Choose $n,m\geq N$, $x\in\mathbb{R}$.  Note that $0<1/n,1/m<\delta/2$ and thus $|1/n-1/m|\leq 1/n+1/m<\delta$ or $|(x-1/m)-(x-1/n)|<\delta$.  Thus $|f_n(x)-f_m(x)|<\epsilon$.  We conclude that the Cauchy criterion is met and thus $f_n(x)$ converges uniformly.  Also note that as $n\rightarrow\infty$, $f(x-1/n)\rightarrow f(x)$.  Thus $f_n\rightarrow f$ point-wise.\\
To the point that uniform continuity is necessary, consider the function $f(x)=x^2$.  This function violates uniform continuity and also will not have the property described above.  This can be demonstrated easily since $|f(x-1/n)-f(x)|=|-2x/n+1/n^2|$ can be made large for any particular $n$ by choosing a large $x$, in other words if you gave me a $N$ that was supposed to work with a $\epsilon$ I could choose a huge $x$ value and break the uniform convergence inequality.
\begin{exercise}

Abbott 6.3.5
\end{exercise}
Define $g_n(x)=\frac{nx+x^2}{2n}$ and $g(x)$ as the limit of the $g_n(x)$.
\begin{enumerate}[(a)]
\item
Note that $g_n(x)=\frac{nx+x^2}{2n}=\frac{x+x^2/n}{2}\rightarrow x/2=g(x)$.  Noting that $x/2$ is a polynomial we can say $g(x)$ is differentiable and $g'(x)=1/2$.
\item
Note that $g_n'(x)=\frac{n+2x}{2n}=\frac{1+2x/n}{2}$.  Consider a interval $[-M,M]$.  Choose $\epsilon>0$.  Note that there exists a $N\in\mathbb{N}$ such that $1/N<\epsilon/2M$.  Choose $n,m\geq N$.  Choose $x\in[-M,M]$.  Note that $|g_n'(x)-g_m'(x)|=|x/n-x/m|\leq |x/n|+|x/m|<M/n+M/m<\epsilon$.  Conclude that $g_n'(x)$ converges uniformly and note that it converges on $1/2$.  Conclude $g'(x)=1/2$.
\item
Define $f_n(x)=\frac{nx^2+1}{2n+x}$.\\
Note that $f_n(x)=\frac{x^2+1/n}{2+x/n}\rightarrow x^2/2=f(x)$, thus $f'(x)=x$.
\item
Note that $f_n'(x)=\frac{4n^2x+2n+2nx^2+x-nx^2+1}{(2n+x)^2}=\frac{4x+2n+x^2/n+x/n^2+1/n^2}{4+4x/n+x^2/n^2}$.
\end{enumerate}











\newpage
\W
\begin{exercise}

Abbott 6.2.5
\end{exercise}
\begin{proof}
Suppose $f_n:A\rightarrow\mathbb{R}$.\\\\
Suppose for every $\epsilon>0$ there exists a $N\in\mathbb{N}$ such that if $n,m\geq N$ and $x\in A$, $|f_n(x)-f_m(x)|<\epsilon$.  Note that for a particular $x$, $f_n(x)$ is a Cauchy sequence and thus converges, thus $f_n(x)$ converges point-wise to some function $f(x)$.  Choose $\epsilon>0$.  There exists a $N\in\mathbb{N}$ such that if $n,m\geq N$ and $x\in A$, $|f_n(x)-f_m(x)|<\epsilon/2$.  Choose $n\geq N$.  Note that $|f_n(x)-f_m(x)|<\epsilon/2$, $f_n(x)-\epsilon/2<f_m(x)<f_n(x)+\epsilon/2$ for all $m\geq N$.  By the limit order theorem $f_n(x)-\epsilon<f_n(x)-\epsilon/2\leq f(x)\leq f_n(x)+\epsilon/2<f_n(x)+\epsilon$, so $|f_n(x)-f(x)|<\epsilon$.  Therefore $f_n\rightarrow f$ uniformly.\\\\
Suppose $f_n\rightarrow f$ uniformly.  Choose $\epsilon>0$.  There exists a $N\in\mathbb{N}$ such that for all $x\in A$ and $n\geq N$, $|f_n(x)-f(x)|<\epsilon/2$.  Choose $x\in A$, $n,m\geq N$.  Note that $|f_n(x)-f_m(x)|=|f_n(x)-f(x)+f(x)-f_m(x)|\leq |f_n(x)-f(x)|+|f(x)-f_m(x)|<\epsilon/2+\epsilon/2=\epsilon$.\\\\
We have now demonstrated that a sequence converges uniformly if and only if it adheres to the Cauchy criterion for uniform convergence.
\end{proof}

\end{document}





























