%%%%%%%%%%%%%%%%%%%%%%%%%%%%%%%%%%%%%%%%%%%%%%%%%%%%%%%%%%%%%%%%%%%%%%%%%%%%%%%%%%%%%%%
%%%%%%%%%%%%%%%%%%%%%%%%%%%%%%%%%%%%%%%%%%%%%%%%%%%%%%%%%%%%%%%%%%%%%%%%%%%%%%%%%%%%%%%
% 
% This top part of the document is called the 'preamble'.  Modify it with caution!
%
% The real document starts below where it says 'The main document starts here'.

\documentclass[12pt]{article}

\usepackage{amssymb,amsmath,amsthm}
\usepackage[top=1in, bottom=1in, left=1.25in, right=1.25in]{geometry}
\usepackage{fancyhdr}
\usepackage{enumerate}
\usepackage{color}

% Comment the following line to use TeX's default font of Computer Modern.
\usepackage{times,txfonts}

\newtheoremstyle{homework}% name of the style to be used
  {18pt}% measure of space to leave above the theorem. E.g.: 3pt
  {12pt}% measure of space to leave below the theorem. E.g.: 3pt
  {}% name of font to use in the body of the theorem
  {}% measure of space to indent
  {\bfseries}% name of head font
  {:}% punctuation between head and body
  {2ex}% space after theorem head; " " = normal interword space
  {}% Manually specify head
\theoremstyle{homework} 

% Set up an Exercise environment and a Solution label.
\newtheorem*{exercisecore}{Exercise \@currentlabel}
\newenvironment{exercise}[1]
{\def\@currentlabel{#1}\exercisecore}
{\endexercisecore}

\newcommand\W{{\color{red}\textbf{(W) (Hand this one in to David.)}}}
\newcommand\tome{{\color{red}\textbf{(Hand this one in to David.)}}}

\newcommand{\localhead}[1]{\par\smallskip\noindent\textbf{#1}\nobreak\\}%
\newcommand\solution{\localhead{Solution:}}

%%%%%%%%%%%%%%%%%%%%%%%%%%%%%%%%%%%%%%%%%%%%%%%%%%%%%%%%%%%%%%%%%%%%%%%%
%
% Stuff for getting the name/document date/title across the header
\makeatletter
\RequirePackage{fancyhdr}
\pagestyle{fancy}
\fancyfoot[C]{\ifnum \value{page} > 1\relax\thepage\fi}
\fancyhead[L]{\ifx\@doclabel\@empty\else\@doclabel\fi}
\fancyhead[C]{\ifx\@docdate\@empty\else\@docdate\fi}
\fancyhead[R]{\ifx\@docauthor\@empty\else\@docauthor\fi}
\headheight 15pt

\def\doclabel#1{\gdef\@doclabel{#1}}
\doclabel{Use {\tt\textbackslash doclabel\{MY LABEL\}}.}
\def\docdate#1{\gdef\@docdate{#1}}
\docdate{Use {\tt\textbackslash docdate\{MY DATE\}}.}
\def\docauthor#1{\gdef\@docauthor{#1}}
\docauthor{Use {\tt\textbackslash docauthor\{MY NAME\}}.}
\makeatother

% Shortcuts for blackboard bold number sets (reals, integers, etc.)
\newcommand{\Reals}{\ensuremath{\mathbb R}}
\newcommand{\Nats}{\ensuremath{\mathbb N}}
\newcommand{\Ints}{\ensuremath{\mathbb Z}}
\newcommand{\Rats}{\ensuremath{\mathbb Q}}
\newcommand{\Cplx}{\ensuremath{\mathbb C}}
%% Some equivalents that some people may prefer.
\let\RR\Reals
\let\NN\Nats
\let\II\Ints
\let\CC\Cplx

%%%%%%%%%%%%%%%%%%%%%%%%%%%%%%%%%%%%%%%%%%%%%%%%%%%%%%%%%%%%%%%%%%%%%%%%%%%%%%%%%%%%%%%
%%%%%%%%%%%%%%%%%%%%%%%%%%%%%%%%%%%%%%%%%%%%%%%%%%%%%%%%%%%%%%%%%%%%%%%%%%%%%%%%%%%%%%%
% 
% The main document start here.

% The following commands set up the material that appears in the header.
\doclabel{Math 401: Final}
\docauthor{Parker Whaley}
\docdate{Due December 15, 2016}

\begin{document}
Note that I am operating under the convention $V_\epsilon(d)=(d-\epsilon,d+\epsilon)$.
\begin{exercise}
1
Suppose $(x_n)$ and $(y_n)$ are sequences such that $\lim_{n\rightarrow \infty} x_n = L$ and $lim_{n\rightarrow\infty} y_n =\infty$.  Show that $lim_{n\rightarrow\infty} x_n/y_n = 0$.
\end{exercise}
\begin{proof}
Suppose $(x_n)$ and $(y_n)$ are sequences such that $\lim_{n\rightarrow \infty} x_n = L$ and $lim_{n\rightarrow\infty} y_n =\infty$.  Note that since $x_n$ is a convergent sequence it must be bounded thus there exists $M\in(0,\infty)$ such that $|x_n|<M$ for all $n\in\mathbb{N}$.  Choose $\epsilon>0$.  Define $k\in\mathbb{N}$ such that $1/k<\epsilon/M$.  Define $N\in\mathbb{N}$ such that $y_n>k$ for all $n\in [N,\infty]\cap \mathbb{N}$.  Choose $n\in\mathbb{N}$ where $n\geq N$.  Note that $|x_n/y_n-0|=|x_n|/|y_n|=|x_n|/y_n<|x_n|/k<M/k<\epsilon$.  We conclude that $x_n/y_n\rightarrow 0$.
\end{proof}

\begin{exercise}
2
A number is algebraic if it is a solution of a polynomial equation $a_nx^n + a_{n−1}x^{n−1} + \cdots + a_1x + a_0 = 0$ where each $a_k$ is an integer, $n \geq 1$, and $a_n \neq 0$. Show that the collection of all algebraic numbers is countable.
\end{exercise}
First let us prove that polynomials have finitely many zeros.
\begin{proof}
Note that a polynomial as described above of order 1 takes the form $P_1(x)=a_1x+a_0$.  Suppose that there were infinitely many solution to $P_1(x)=0$, choose two of these, $x_1<x_2$.  Note that $P'_1(x)=a_1\neq 0$ for all $x\in\mathbb{R}$.  By the mean value theorem there exists some $x_3$ such that $x_3\in(x_1,x_2)$ and $P'_1(x_3)=0$.  This is a contradiction and thus we conclude that $P_1(x)$ has finitely many solutions.\\\\
Suppose that all polynomials of degree $n-1$ have finitely many solutions.  Consider $P_n(x)=\sum_{k=0}^n a_k x^k$ a arbitrary polynomial of degree $n$.  Note that $P'_n(x)=\sum_{k=1}^n ka_k x^{k-1}$ a polynomial of degree $n-1$.  We conclude that $P'_n(x)=0$ has finitely many solutions, define $l$ to be the number of solutions to $P'_n(x)=0$.  Suppose $P_n(x)=0$ has infinitely many solutions.  Select $l+2$ solutions to $P_n(x)=0$ and arrange them in a list $\{x_k\}_{k=0}^{l+1}$ such that $x_k<x_{k+1}$ for all $k\in [0,l+1]\cap\mathbb{N}$.  Now construct intervals $I_k=(x_{k-1},x_k)$ for all $k\in [1,l+1]\cap\mathbb{N}$.  Note that there are no shared elements between any two intervals.  By the mean value theorem there exists a value $y_k\in I_k$ such that $P'(y_k)=0$.  Noting that there are $l+1$ non-identical $y_k$'s we conclude that we have found $l+1$ solutions to $P'_n(x)=0$, a contradiction, we are forced to conclude that there are finitely many solution to $P_n(x)=0$.\\\\
By induction conclude that there are finitely many solutions to any polynomial as constructed above.
\end{proof}
Now to prove that there are countably infinite algebraic numbers.
\begin{proof}
Recall from proofs that each natural number has a unique prime factorization and also that there are countably infinite primes.  Now we can define a bijective mapping from a polynomial of the described form to the naturals.  Define $p_n$ to be the $n$'th prime number.  Associate a arbitrary polynomial $P_n(x)=\sum_{k=0}^N a_k x^k$ where $N\in\mathbb{N}$ with the natural $\sum^N_{k=1}p_n^{a_n}$.  Note that each polynomial is associated with one and only one natural, via the uniqueness of prime factorization.  Note that given a natural number there is at most one polynomial associated with it, via the uniqueness of prime factorization, thus this mapping is one-to-one.  Call this mapping $M_1$ witch takes a polynomial to a natural in a one-to-one manner.  Note that there exists a bijective map from $\mathbb{N}\times \mathbb{N}\rightarrow\mathbb{N}$ call this map $M_2$.  Consider a arbitrary algebraic number $k$.  Note that there exists a non-empty set of polynomials $S$ where $P(k)=0$ for all $P(x)\in S$.  Define $P(x)$ to be the polynomial in $S$ with the smallest mapped value under $M_1$.  Take $X=\{x\in\mathbb{R}:P(x)=0\}$.  Note that there are finitely many elements in $X$ as proved in the previous proof.  Ordering $X$ by increasing value we can associate $k$ with a integer $i$ where $i$ is $k$'s position in the ordered $X$.  Note that I have now defined a mapping, let's call it $M_3$ that maps a algebraic number to a polynomial and a natural number.  Note that $M_3$ is one-to-one, given a polynomial $P(x)$ a a natural $n$ there is at most one number that is the $n$'th solution to $P(x)=0$.  Note that now I can construct a one-to-one mapping from the algebraic numbers to the naturals, use $M_3$ to map a algebraic to a polynomial and a natural, use $M_1$ to map the polynomial to a natural, use $M_2$ to map the resulting two naturals to one natural, since all of these steps are individually one-to-one the entire process is one-to-one.  Now we can conclude that the algebraic numbers are at most countably infinite.  Noting that for every $n\in\mathbb{N}$, $n$ will be a solution to $x-n=0$ we conclude the naturals are algebraic and thus that the algebraic numbers are at least countably infinite.  We are forced to conclude that the algebraic numbers are countably infinite.
\end{proof}
\begin{exercise}
3
Let $p$ be a fifth order polynomial, so $p(x) = \sum^5_{k=0} a_kx^k$ where each $a_k\in\mathbb{R}$ , and $a_5 \neq 0$.
Prove that there is a solution of $p(x) = 0$.
\end{exercise}
\begin{proof}
Suppose $p(x) = \sum^5_{k=0} a_kx^k$ where each $a_k\in\mathbb{R}$ , and $a_5 \neq 0$.  Suppose $a_5>0$.  Define $a=a_5/5$.  Note that I can now re-write $p(x) = \sum^5_{k=0} a_kx^k=5ax^5+\sum^4_{k=0} a_kx^k=\sum^4_{k=0} ax^5+a_kx^k$.  Note that $a>0$.  Choose some $x_M>\max(1,\{|a_n|/a:n\in\{0,1,2,3,4\}\})$.  Note that $ax_M^5=ax_Mx_M^4>|a_n|x_M^4=|a_nx_M^n||x_M^{4-n}|\geq |a_nx_M^n|\geq -a_nx_M^n$, or $ax_M^5+a_nx_M^n>0$ for all $n\in\{0,1,2,3,4\}$.  Now note that $p(x_M) =\sum^4_{k=0} ax_M^5+a_kx_M^k>0$.  Choose some $x_m<\min(-1,\{-|a_n|/a:n\in\{0,1,2,3,4\}\})$.  Note that $ax_m^5=ax_mx_m^4<-|a_n|x_m^4=-|a_nx_m^4|=-|a_nx_m^n||x_m^{4-n}|\leq -|a_nx_m^n|\leq -a_nx_m^n$, or $ax_m^5+a_nx_m^n<0$ for all $n\in\{0,1,2,3,4\}$.  Now note that $p(x_m) =\sum^4_{k=0} ax_m^5+a_kx_m^k<0$.  By the intermediate value theorem there exists some $x'\in[x_m,x_M]$ such that $p(x')=0$.  We now conclude that there is a solution to a fifth order polynomial with $a_5>0$.\\\
Suppose $p(x) = \sum^5_{k=0} a_kx^k$ where each $a_k\in\mathbb{R}$ , and $a_5 \neq 0$.  Suppose $a_5<0$.  Note that $p_2(x)=-p(x)$ is a fifth order polynomial with $a_5>0$ and thus there exist some $x'$ such that $p_2(x')=0$, note that $p(x')=-p_2(x')=0$.  We now can conclude that there is a solution to $p(x) = 0$ for any fifth order polynomial, $a_5 \neq 0$.
\end{proof}
\begin{exercise}
4
We say that a function $f : \mathbb{R}\rightarrow\mathbb{R}$ is periodic if there is a number $L$ such that $f (x) =
f (x+L)$ for all $x \in\mathbb{R}$. Show that a continuous, periodic function is uniformly continuous.
\end{exercise}
\begin{proof}
Suppose $f : \mathbb{R}\rightarrow\mathbb{R}$ is periodic with periodicity $L$ and continuous.  Define $A=[-L,L]$.  Note that $A$ is closed and bounded thus compact.  Note that $f$ is uniformly convergent on $A$.  Choose $\epsilon>0$.  Note that there exists $\delta>0$ such that if $x,y\in A$ and $|x-y|<\delta$ then $|f(x)-f(y)|<\epsilon$.  Choose $x,y\in\mathbb{R}$ such that $|x-y|<\min(\delta,L)$.  Without loss of generality suppose $x\geq y$.\\\\
Suppose $x>0$.  Define sequences $x_0=x$ and $y_0=y$ and $x_n=x_{n-1}-L$ and $y_n=y_{n-1}-L$ for $n\in\mathbb{N}$.  Note that $|f(x_n)-f(y_n)|=|f(x_{n-1})-f(y_{n-1})|$ and thus $|f(x_n)-f(y_n)|=|f(x_0)-f(y_0)|=|f(x)-f(y)|$  Define $n'\in\mathbb{Z}$ to be the floor of $x/L$.  Note that $x/L\geq n'>x/L-1\geq -1$, thus $n'\in\mathbb{N}+\{0\}$.  Note that $x\geq Ln'>x-L$ and $0\geq Ln'-x>-L$ and $0\leq x-Ln'<L$.  Note that $x-Ln'=x_{n'}$.  Note that $x_{n'}-y_{n'}=x-y=|x-y|<L$ thus $-L\leq x_{n'}-L<y_{n'}<x_{n'}<L$.  Thus $x_{n'},y_{n'}\in A$ and $|x_{n'}-y_{n'}|=|x-y|<\delta$ so $|f(x)-f(y)|=|f(x_{n'})-f(y_{n'})|<\epsilon$.\\\\
The exercise for this in the case the $x<0$ is a copy paste with a few sign changes and inequality flips (I was instructed on the midterm to refrain from doing this repetition) however the conclusion is the same.  This allows us to say $|f(x)-f(y)|<\epsilon$.  We conclude that the function is uniform convergent.
\end{proof}
\begin{exercise}
5
Use the Nested Interval Property to deduce the Axiom of Completeness without using any other form of the Axiom of Completeness.  Hint: Look at the proof of the Bolzano Weierstrass Theorem.
\end{exercise}
\begin{proof}
Suppose $S$ is a bounded set.  There exists $M$ such that $x<M$ for all $x\in S$.  There exists $x\in S$.  Define a sequence by $a_0=x-1$ and 
$$a_n=\begin{cases} a_{n-1} & \frac{a_{n-1}+b_{n-1}}{2} \text{ is a upper bound on }S\\
\frac{a_{n-1}+b_{n-1}}{2} & \text{otherwise}\end{cases}$$for all $n\in\mathbb{N}$.  
Define a sequence by $b_0=M$ and 
$$b_n=\begin{cases} \frac{a_{n-1}+b_{n-1}}{2} & \frac{a_{n-1}+b_{n-1}}{2} \text{ is a upper bound on }S\\
b_{n-1} & \text{otherwise}\end{cases}$$for all $n\in\mathbb{N}$.  
Note that $b_n-a_n=\frac{b_0-a_0}{2^n}\leq \frac{b_0-a_0}{n}$ for all $n\in\mathbb{N}$.  Note that $a_n$ is not a upper bound on $S$ for all $n\in\mathbb{N}$.  Note that $b_n$ is a upper bound on $S$ for all $n\in\mathbb{N}$.  Consider the sequence of intervals $I_n=[a_n,b_n]$.  Note that $I_n\subseteq I_{n-1}$ for all $n\in\mathbb{N}$.  By NIP we can conclude that $K=\cap_{n=0}^\infty I_n\neq \emptyset$.  Suppose that $a\neq b\in K$.  Note that there exists a $n_1\in\mathbb{N}$ such that $\frac{b_0-a_0}{n_1}<|b-a|$.  Note that $a,b\in I_{n_1}$ and thus $|b-a|\leq b_{n_1}-a_{n_1}<\frac{b_0-a_0}{{n_1}}$, a contradiction conclude that there is one and only one element in $K$, call this number $c$.  Suppose that there exists $d\in S$ such that $c<d$.  Note that there exists a $n_2\in\mathbb{N}$ such that $\frac{b_0-a_0}{n_2}<|d-c|$.  Note that $a_{n_2}\leq c<d\leq b_{n_2}$ and thus $|c-d|\leq b_{n_2}-a_{n_2}<\frac{b_0-a_0}{{n_2}}$, a contradiction conclude that $c$ is a upper bound on $S$.  Suppose that $e$ is a upper bound on $S$ and $e<c$.  Note that there exists a $n_3\in\mathbb{N}$ such that $\frac{b_0-a_0}{n_3}<|e-c|$.  Note that $a_{n_3}\leq e<c\leq b_{n_3}$ and thus $|c-e|\leq b_{n_3}-a_{n_3}<\frac{b_0-a_0}{{n_3}}$, a contradiction conclude that $c$ is less than or equal to every upper bound on $S$.  We have proven $c=\sup(S)$, thus we conclude AoC.
\end{proof}
\begin{exercise}
6
Let $A$ be a subset of $\mathbb{R}$. The closure of $A$ is the union of $A$ together with its limit points.
We denote it by $\bar{A}$.
\end{exercise}
\begin{enumerate}[(a)]
\item
Explain why it is not obvious that $\bar{A}$ is a closed set.\\
It is non-obvious since we are not guaranteed that $\bar{A}$ has the same limit points as $A$.  Perhaps adding in all of $A$'s limit points generated a new limit point, witch would not be included in $\bar{A}$.
\item
Suppose $c$ is a limit point of $A$. Show that for every $\epsilon > 0$ there are at least two
points $x\in A$ with $x\neq c$ and $|x - c| <\epsilon$.\\
By the definition of limit point $V_\epsilon(c)\cap A- \{c\}\neq \emptyset$.  Choose one on these elements and call it $x_1$.  Note that $x_1\neq c$ and $|x_1 - c| <\epsilon$.  Define $\epsilon_2=|x_1 - c|$.  By the definition of limit point $V_{\epsilon_2}(c)\cap A- \{c\}\neq \emptyset$.  Choose one on these elements and call it $x_2$.  Note that $x_2\neq c$ and $x_2\neq x_1$ and $|x_2 - c| <\epsilon_2<\epsilon$.
\item
Suppose $d$ is a limit point of $\bar{A}$. Show that there exists a sequence in $A$, with no
terms equal to $c$, that converges to $d$.\\
Suppose $d$ is not a limit point of $A$.  There must exist $\epsilon>0$ such that $V_\epsilon(d)\cap A-\{d\}=\emptyset$.  Note that since $d$ is a limit point of $\bar{A}$, $V_{\epsilon/2}\cap \bar{A}-\{d\}\neq\emptyset$.  Let $a\in V_{\epsilon/2}\cap \bar{A}-\{d\}$.  Define $\delta=\min(|a-d|,\epsilon/2)$.  Suppose $x\in V_\delta(a)\cap A-\{a\}$.  Note that $x\neq d$ since $|a-x|<\delta\leq |a-d|$.  Note that $|a-d|<\epsilon/2$ also note that $|x-d|=|x-a+a-d|\leq |x-a|+|a-d|<\epsilon/2+\epsilon/2$ thus $x\in V_\epsilon(d)$.  Note that $x\in V_\epsilon(d)\cap A-\{d\}=\emptyset$ we can now conclude that no such $x$ could exist and thus $V_\delta(a)\cap A-\{a\}=\emptyset$.  This means that $a$ is not a limit point of $A$ and thus since it ended up in $\bar{A}$ we can conclude $a\in A$.  Thus $a\in V_\epsilon(d)\cap A-\{d\}=\emptyset$, a contradiction, thus $d$ is a limit point of $A$.  Since $d$ is a limit point of $A$ it must also be a limit point of $A-\{c\}$ thus there exists a sequence in $A-\{c\}$ converging on $d$.
\item
Conclude that $\bar{A}$ is closed.\\
Suppose $d$ is a limit point of $\bar{A}$.  Note that $d$ is a limit point of $A$.  Thus $d\in\bar{A}$.
\end{enumerate}
\begin{exercise}
7
Let $(r_n)$ be an enumeration of the rational numbers. Define $f:\RR \to \RR$ by $$f(x) =
\begin{cases}
1/n & x=r_n \\
0 & x\neq\mathbb{Q}
\end{cases}$$
Determine, with proof, where $f$ is continuous.
\end{exercise}
\begin{proof}
Choose $x\in\mathbb{Q}$.  Suppose $f$ is continuous at $x$.  Define $n\in\mathbb{N}$ such that $r_n=x$.  Define $\epsilon=1/n$.  There exists $\delta>0$ such that if $y\in V_\delta(x)$ then $|f(x)-f(y)|<\epsilon$.  Define $y\not\in\mathbb{Q}$ such that $y\in V_\delta(x)$.  Note that $|f(x)-f(y)|=|f(x)|=1/n=\epsilon<\epsilon$, a contradiction.  We are forced to conclude that $f$ is discontinuous at every rational.\\
Choose $x\not\in\mathbb{Q}$.  Choose $\epsilon>0$.  Define $n\in\mathbb{N}$ such that $1/n<\epsilon$.  Define $\delta=\min(\{|x-r_k|:k\in\mathbb{N}\text{ and } k\leq n\})$.  Note that $\delta>0$.  Choose $y\in V_\delta(x)$.  Note that $y\not\in \{|x-r_k|:k\in\mathbb{N}\text{ and } k\leq n\}$, thus $f(y)<1/n$.  Note that $|f(x)-f(y)|=|f(y)|<1/n<\epsilon$.  Conclude that $f$ is continuous on the irrationals only.
\end{proof}
\begin{exercise}
8
Abbott 5.3.5 a)
\end{exercise}
Prove the generalized mean value theorem.
\begin{proof}
Suppose $f$ and $g$ are continuous on $[a,b]$ and differentiable on $(a,b)$.  Define $h(x)=[f(b)-f(a)]g(x)-[g(b)-g(a)]f(x)$ where $h(x)$ is defined on $[a,b]$.  By the algebraic theorem for continuity and differentiability we can conclude that $h(x)$ is continuous on $[a,b]$ and differentiable on $(a,b)$.  Note that $h'(x)=[f(b)-f(a)]g'(x)-[g(b)-g(a)]f'(x)$.  By the mean value theorem there exists $c\in(a,b)$ such that $h'(c)=\frac{h(a)-h(b)}{a-b}$.  Note that $[f(b)-f(a)]g'(c)-[g(b)-g(a)]f'(c)=\frac{[f(b)-f(a)]g(a)-[g(b)-g(a)]f(a)-[f(b)-f(a)]g(b)-[g(b)-g(a)]f(b)}{a-b}=\frac{f(b)g(a)-f(a)g(a)-f(a)g(b)+f(a)g(a)-f(b)g(b)+f(a)g(b)+f(b)g(b)-f(b)g(a)}{a-b}=\frac{0}{a-b}=0$.  Now we see that $[f(b)-f(a)]g'(c)-[g(b)-g(a)]f'(c)=0$, and so $[f(b)-f(a)]g'(c)=[g(b)-g(a)]f'(c)$.  We have now concluded the generalized mean value theorem.\\
Suppose further that $g'(x)\neq 0$ for all $x\in(a,b)$.  If $g(a)=g(b)$ we would conclude via the mean value theorem that for some $x\in(a,b)$, $g'(x)=0$ we now can conclude $[g(b)-g(a)]\neq 0$.  Now we can see in this case $[f(b)-f(a)]g'(c)=[g(b)-g(a)]f'(c)$ becomes $[f(b)-f(a)]/[g(b)-g(a)]=f'(c)/g'(c)$ for some $c\in(a,b)$.
\end{proof}
\begin{exercise}
9
Consider the function $$f(x) = \sum_{k=1}^\infty \frac{1}{2^k} sin(kx).$$ Show that $f$ is infinitely differentiable.
\end{exercise}

First recall that the derivative of $\sin(x)$ is $\cos(x)$ and the derivative of$\cos(x)$ is $-\sin(x)$.\\
Next as a sub-proof let me show that $\sum_{k=1}^n \frac{k^\alpha}{2^k}$ converges for any $\alpha\in\mathbb{N}+\{0\}$.
\begin{proof}
Let us apply the ratio test to test for convergence.  Note that $|\frac{2^k(k+1)^\alpha}{2^{k+1}k^\alpha}|=1/2\frac{(k+1)^\alpha}{k^\alpha}$.  Note that $\lim_{k\rightarrow\infty}\frac{(k+1)^\alpha}{k^\alpha}=\lim_{k\rightarrow\infty}\frac{(k+1)^{\alpha-1}}{k^{\alpha-1}}$ if $\alpha\neq 0$, noting that $\alpha\in \mathbb{N}+\{0\}$ we can continue doing this reduction $\alpha$ times to obtain $\lim_{k\rightarrow\infty}\frac{(k+1)^\alpha}{k^\alpha}=\lim_{k\rightarrow\infty}\frac{(k+1)^0}{k^0}=1$.  Thus $\lim_{k\rightarrow\infty}|\frac{2^k(k+1)^\alpha}{2^{k+1}k^\alpha}|=1/2<1$ and so we conclude convergence.
\end{proof}
Now we are ready to begin the proof proper.
\begin{proof}
Define $g_{k,\alpha}(x)=\frac{k^\alpha}{2^k}\sin(kx)$.  Define $f_{n,\alpha}(x)=\sum_{k=1}^n g_{k,\alpha}(x)$.  Note that $|g_{k,\alpha}(x)|=|\frac{k^\alpha}{2^k}\sin(kx)|\leq \frac{k^\alpha}{2^k}$, By the Weierstrass M-Test we can conclude that $f_{n,\alpha}(x)\rightarrow f_\alpha(x)=\sum_{k=1}^\infty g_{k,\alpha}(x)$ uniformly.\\
Repeating this argument with $\cos$ we get exactly the same result.\\
We will begin a proof by induction, our induction hypothesis is the $\alpha$ derivative of $f(x)$ is $$f^{(\alpha)}(x)=\begin{cases}
\sum_{k=1}^\infty \frac{k^\alpha}{2^k}\sin(kx) & 4\alpha\in\mathbb{N}+\{0\}\\
\sum_{k=1}^\infty \frac{k^\alpha}{2^k}\cos(kx) & 4\alpha+1 \in\mathbb{N}+\{0\}\\
\sum_{k=1}^\infty -\frac{k^\alpha}{2^k}\sin(kx) & 4\alpha+2 \in\mathbb{N}+\{0\}\\
\sum_{k=1}^\infty -\frac{k^\alpha}{2^k}\cos(kx) & 4\alpha+3 \in\mathbb{N}+\{0\}\\
\end{cases}$$
In the base case this is true by inspection since the $f^{(0)}$ is the same as the $f$ given in the problem.\\
For the induction step suppose the hypothesis in the $\alpha$ case.  Note that $$f^{(\alpha)}_n(x)=\begin{cases}
\sum_{k=1}^n \frac{k^\alpha}{2^k}\sin(kx) & 4\alpha\in\mathbb{N}+\{0\}\\
\sum_{k=1}^n \frac{k^\alpha}{2^k}\cos(kx) & 4\alpha+1 \in\mathbb{N}+\{0\}\\
\sum_{k=1}^n -\frac{k^\alpha}{2^k}\sin(kx) & 4\alpha+2 \in\mathbb{N}+\{0\}\\
\sum_{k=1}^n -\frac{k^\alpha}{2^k}\cos(kx) & 4\alpha+3 \in\mathbb{N}+\{0\}\\
\end{cases}$$
converges uniformly to $f^{(\alpha)}(x)$ by the first part of this proof.  Note that $${f^{(\alpha)}}'_n(x)=\begin{cases}
\sum_{k=1}^n -\frac{k^{\alpha+1}}{2^k}\cos(kx) & 4\alpha\in\mathbb{N}+\{0\}\\
\sum_{k=1}^n \frac{k^{\alpha+1}}{2^k}\sin(kx) & 4\alpha+1 \in\mathbb{N}+\{0\}\\
\sum_{k=1}^n \frac{k^{\alpha+1}}{2^k}\cos(kx) & 4\alpha+2 \in\mathbb{N}+\{0\}\\
\sum_{k=1}^n -\frac{k^{\alpha+1}}{2^k}\sin(kx) & 4\alpha+3 \in\mathbb{N}+\{0\}\\
\end{cases}$$
converges uniformly on $$h(x)=\begin{cases}
\sum_{k=1}^\infty -\frac{k^{\alpha+1}}{2^k}\cos(kx) & 4\alpha\in\mathbb{N}+\{0\}\\
\sum_{k=1}^\infty \frac{k^{\alpha+1}}{2^k}\sin(kx) & 4\alpha+1 \in\mathbb{N}+\{0\}\\
\sum_{k=1}^\infty \frac{k^{\alpha+1}}{2^k}\cos(kx) & 4\alpha+2 \in\mathbb{N}+\{0\}\\
\sum_{k=1}^\infty -\frac{k^{\alpha+1}}{2^k}\sin(kx) & 4\alpha+3 \in\mathbb{N}+\{0\}\\
\end{cases}$$
so we can conclude that ${f^{(\alpha+1)}}(x)={f^{(\alpha)}}'(x)=h(x)$.  By doing a change of variable $\alpha'=\alpha+1$ we see $$f^{(\alpha')}_n(x)=\begin{cases}
\sum_{k=1}^n \frac{k^{\alpha'}}{2^k}\sin(kx) & 4\alpha'\in\mathbb{N}+\{0\}\\
\sum_{k=1}^n \frac{k^{\alpha'}}{2^k}\cos(kx) & 4\alpha'+1 \in\mathbb{N}+\{0\}\\
\sum_{k=1}^n -\frac{k^{\alpha'}}{2^k}\sin(kx) & 4\alpha'+2 \in\mathbb{N}+\{0\}\\
\sum_{k=1}^n -\frac{k^{\alpha'}}{2^k}\cos(kx) & 4\alpha'+3 \in\mathbb{N}+\{0\}\\
\end{cases}$$
Via induction we now conclude that the $n$th derivative of $f$ is the $f^{(n)}(x)$ described above, and thus infinitely differentiable.
\end{proof}
\begin{exercise}
{10}
Suppose $f:(0,\infty)\to \RR$ is twice differentiable.
\end{exercise}
\begin{enumerate}[(a)]
\item Suppose $|f(x)|\le M_0$ and $|f''(x)|\le M_1$ for all $x>0$. Show that for all $h>0$, $$|f'(x)|\le \frac{2}{h}M_0 + \frac{h}{2}M_1.$$\\
Choose $h>0$, $x>0$.  Note that the Taylor remainder theorem tells us that $f(x+h)=f(x)+f'(x)h+f''(\xi)h^2/2$ for some $\xi$ in the interval between $x$ and $h$.  After some manipulation we can see $|f'(x)|=|f(x+h)/h-f(x)/h-f''(\xi)h/2|\leq M_0/h+M_0/h+M_1h/2=\frac{2}{h}M_0 + \frac{h}{2}M_1$.
\item Suppose $|f(x)|\le M_0$ and $|f''(x)|\le M_1$ for all $x>0$. Show that for all $x>0$, $$|f'(x)|\le 2\sqrt{M_0M_1}.$$\\
Suppose $M_1=0$, in this case $f''(x)=0$ for all $x$, this means that $f'(x)$ is a constant, and $f(x)=f(0)+f'(x)h$, notice that in order for $f(x)$ to be bounded $f'(x)=0$.  In this case our inequality becomes $0=|f'(x)|\le 2\sqrt{M_0M_1}=0$, clearly true.\\
Suppose $M_0=0$, in this case we have a constant function and thus $f'(x)=0$.  In this case our inequality becomes $0=|f'(x)|\le 2\sqrt{M_0M_1}=0$, clearly true.\\
Suppose $M_1\neq 0$ and $M_2\neq 0$.  In this case we can define $h=2\sqrt{M_0}/\sqrt{M_1}>0$.  Notice that the first relation becomes $|f'(x)|\leq \frac{2}{h}M_0 + \frac{h}{2}M_1=\sqrt{M_0M_1} + \sqrt{M_0M_1}=2\sqrt{M_0M_1}$.
\item Suppose instead that $f''(x)$ is bounded for $x>0$ and $\lim_{x\to\infty} f(x)=0$. Show $\lim_{x\to\infty}|f'(x)|=0$.\\
Define $M_1$ to be a upper bound on $f''(x)$.  Choose $\epsilon>0$.  Define $M$ such that if $x>M$, $f(x)<\epsilon^2/(4M_1)$.  Note that the previous two proofs did not utilize the starting at zero property of $f$, thus they hold just as well for the section of $f: (M,\infty)$.  In this range $|f(x)|<M_0=\epsilon^2/(4M_1)$ and thus $|f'(x)|\le 2\sqrt{M_0M_1}<\epsilon$.  We can now conclude that $\lim_{x\to\infty}|f'(x)|=0$.
\end{enumerate}
\begin{exercise}
{11}
Let $g_n$ and $g$ be uniformly bounded on $[0,1]$, meaning that there exists a single $M>0$ satisfying $|g(x)|\le M$ and $|g_n(x)|\le M$ for all $n \in \NN$ and $x \in [0,1]$. Assume $g_n \to g$ point-wise on $[0,1]$ and uniformly on any set of the form $[0,\alpha]$, where $0 < \alpha<1$.

If all the functions are integrable, show that $\lim_{n\to\infty}\int_0^1 g_n = \int_0^1 g$.
\end{exercise}

\begin{proof}
Choose $\epsilon>0$.  Define $\alpha=\max(1-\epsilon/(2M),1/2)$.  Note that $\lim_{n\to\infty}\int_0^\alpha g_n=\int_0^\alpha g$, by the integrable limit theorem.  Define $N\in\mathbb{N}$ such that if $n\geq N$, $|\int_0^\alpha g_n-\int_0^\alpha g|<\epsilon/2$.  Choose $n\geq N$.  Note that $|\int_\alpha^1 g_n|\leq M(1-\alpha)$ by Theorem 7.4.2, and $|\int_\alpha^1 g|\leq M(1-\alpha)$.  Note that $|\int_0^1 g_n-\int_0^1 g|\leq |\int_0^\alpha g_n-\int_0^\alpha g|+|\int_\alpha^1 g_n|+|\int_\alpha^1 g|<\epsilon/2+2M(1-\alpha)\leq\epsilon$.
\end{proof}


\begin{exercise}{12} Given a function $f$ on $[a,b]$, define the $total$ $variation$ of $f$ to be $$Vf = \sup \left \{ \sum_{k=1}^n |f(x_k) - f(x_{k-1})| \right \} ,$$ where the supremum is taken over all partitions $P$ of $[a,b]$.
\end{exercise}
\begin{enumerate}[(a)]
\item If $f$ is continuously differentiable ($f'$ exists as a continuous function), use the Fundamental Theorem of Calculus to show $Vf \le \int_a^b|f'|$.\\
Consider a arbitrary partition of $[a,b]$ call it $P$.  Note that $\int_a^b|f'|=\sum_{k=1}^n \int_{x_{k-1}}^{x_k}|f'|\geq \sum_{k=1}^n |\int_{x_{k-1}}^{x_k}f'|=\sum_{k=1}^n |f(x_k) - f(x_{k-1})|$.  Note that $\int_a^b|f'|$ is a upper bound on the set that $Vf$ is taking the supremum of, thus $Vf \le \int_a^b|f'|$.
\item Use the Mean Value Theorem to establish the reverse inequality and conclude that $Vf = \int_a^b|f'|$.\\
Take $P$ to be the partition using the extrema of the function $f$.  Note that $\int_a^b|f'|=\sum_{k=1}^n \int_{x_{k-1}}^{x_k}|f'|= \sum_{k=1}^n |\int_{x_{k-1}}^{x_k}f'|=\sum_{k=1}^n |f(x_k) - f(x_{k-1})|$.
\end{enumerate}








\end{document}