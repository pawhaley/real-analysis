%%%%%%%%%%%%%%%%%%%%%%%%%%%%%%%%%%%%%%%%%%%%%%%%%%%%%%%%%%%%%%%%%%%%%%%%%%%%%%%%%%%%%%%
%%%%%%%%%%%%%%%%%%%%%%%%%%%%%%%%%%%%%%%%%%%%%%%%%%%%%%%%%%%%%%%%%%%%%%%%%%%%%%%%%%%%%%%
% 
% This top part of the document is called the 'preamble'.  Modify it with caution!
%
% The real document starts below where it says 'The main document starts here'.

\documentclass[12pt]{article}

\usepackage{amssymb,amsmath,amsthm}
\usepackage[top=1in, bottom=1in, left=1.25in, right=1.25in]{geometry}
\usepackage{fancyhdr}
\usepackage{enumerate}
\usepackage{color}

% Comment the following line to use TeX's default font of Computer Modern.
\usepackage{times,txfonts}

\newtheoremstyle{homework}% name of the style to be used
  {18pt}% measure of space to leave above the theorem. E.g.: 3pt
  {12pt}% measure of space to leave below the theorem. E.g.: 3pt
  {}% name of font to use in the body of the theorem
  {}% measure of space to indent
  {\bfseries}% name of head font
  {:}% punctuation between head and body
  {2ex}% space after theorem head; " " = normal interword space
  {}% Manually specify head
\theoremstyle{homework} 

% Set up an Exercise environment and a Solution label.
\newtheorem*{exercisecore}{Exercise \@currentlabel}
\newenvironment{exercise}[1]
{\def\@currentlabel{#1}\exercisecore}
{\endexercisecore}

\newcommand\W{{\color{red}\textbf{(W) (Hand this one in to David.)}}}
\newcommand\tome{{\color{red}\textbf{(Hand this one in to David.)}}}

\newcommand{\localhead}[1]{\par\smallskip\noindent\textbf{#1}\nobreak\\}%
\newcommand\solution{\localhead{Solution:}}

%%%%%%%%%%%%%%%%%%%%%%%%%%%%%%%%%%%%%%%%%%%%%%%%%%%%%%%%%%%%%%%%%%%%%%%%
%
% Stuff for getting the name/document date/title across the header
\makeatletter
\RequirePackage{fancyhdr}
\pagestyle{fancy}
\fancyfoot[C]{\ifnum \value{page} > 1\relax\thepage\fi}
\fancyhead[L]{\ifx\@doclabel\@empty\else\@doclabel\fi}
\fancyhead[C]{\ifx\@docdate\@empty\else\@docdate\fi}
\fancyhead[R]{\ifx\@docauthor\@empty\else\@docauthor\fi}
\headheight 15pt

\def\doclabel#1{\gdef\@doclabel{#1}}
\doclabel{Use {\tt\textbackslash doclabel\{MY LABEL\}}.}
\def\docdate#1{\gdef\@docdate{#1}}
\docdate{Use {\tt\textbackslash docdate\{MY DATE\}}.}
\def\docauthor#1{\gdef\@docauthor{#1}}
\docauthor{Use {\tt\textbackslash docauthor\{MY NAME\}}.}
\makeatother

% Shortcuts for blackboard bold number sets (reals, integers, etc.)
\newcommand{\Reals}{\ensuremath{\mathbb R}}
\newcommand{\Nats}{\ensuremath{\mathbb N}}
\newcommand{\Ints}{\ensuremath{\mathbb Z}}
\newcommand{\Rats}{\ensuremath{\mathbb Q}}
\newcommand{\Cplx}{\ensuremath{\mathbb C}}
%% Some equivalents that some people may prefer.
\let\RR\Reals
\let\NN\Nats
\let\II\Ints
\let\CC\Cplx

%%%%%%%%%%%%%%%%%%%%%%%%%%%%%%%%%%%%%%%%%%%%%%%%%%%%%%%%%%%%%%%%%%%%%%%%%%%%%%%%%%%%%%%
%%%%%%%%%%%%%%%%%%%%%%%%%%%%%%%%%%%%%%%%%%%%%%%%%%%%%%%%%%%%%%%%%%%%%%%%%%%%%%%%%%%%%%%
% 
% The main document start here.

% The following commands set up the material that appears in the header.
\doclabel{Math 401: Homework 6}
\docauthor{Parker Whaley}
\docdate{Due October 26, 2016}

\begin{document}
Note that I am operating under the convention that $N,n,m,i,j$ are natural numbers unless otherwise specified.  I am also operating under the convention $v_a(b)=\{x\in\mathbb{R}:b-a<x<b+a\}$
\begin{exercise}

Prove the following
\end{exercise}
\begin{enumerate}[(a)]
\item
$\lim_{x\rightarrow 2} (3x+4)=10$
\begin{proof}
Choose $\epsilon>0$.  Define $\delta=\epsilon/3>0$.  Choose a $x$ such that $0<|x-2|<\delta$.  Note that $-\delta<x-2<\delta$ so $2-\delta<x<2+\delta$ or $6-3\delta<3x<6+3\delta$ so $10-\epsilon<3x+4<10+\epsilon$ or $|(3x+4)-10|<\epsilon$.
\end{proof}
\item
$\lim_{x\rightarrow 0} x^3=0$
\begin{proof}
Choose $\epsilon>0$.  Define $\delta^3=\epsilon$.  Note that $\delta>0$.  Choose a $x$ such that $0<|x|<\delta$.  Note that $-\delta<x<\delta$ so $-\delta^3<x^3<\delta^3$ or $|x^3|<\epsilon$.
\end{proof}
\end{enumerate}
\begin{exercise}

4.2.1 a,b
\end{exercise}
\begin{enumerate}[(a)]
\item
Show how Corollary 4.2.4 (ii) follows from the sequential criterion for limits in therm 4.2.3 and the algebraic limit therm.\\
\begin{proof}
Suppose as $x\rightarrow c$, $f(x)\rightarrow L$ and $g(x)\rightarrow M$.  Where $f$ and $g$ have domain $A$.\\
Choose a sequence $a_n$ where $a_n\in A-\{c\}$ and $a_n\rightarrow c$.  Define $h(x)=f(x)+g(x)$ and define the sequences $f_n=f(a_n)$, $g_n=g(a_n)$, and $h_n=h(a_n)$.  Note that $g_n\rightarrow M$ and $f_n\rightarrow L$.  Note that $h_n=g_n+f_n$.  by the arithmetic limit therm $h_n\rightarrow L+M$.  Since $a_n$ was chosen arbitrarily we can say that $h(x)\rightarrow L+M$ and $x\rightarrow c$.
\end{proof}
\item
Prove again from definition.
\begin{proof}
Suppose as $x\rightarrow c$, $f(x)\rightarrow L$ and $g(x)\rightarrow M$.  Where $f$ and $g$ have domain $A$.\\
Choose $\epsilon>0$.  There must exist $\delta_1>0$ such that for all $0<|x-c|<\delta_1$, $|f(x)-L|<\epsilon/2$.  There must exist $\delta_2>0$ such that for all $0<|x-c|<\delta_2$, $|g(x)-M|<\epsilon/2$.  Define $h(x)=f(x)+g(x)$.  Define $\delta=\min(\delta_1,\delta_2)$.  Choose $0<|x-c|<\delta$.  Note that $0<|x-c|<\delta_1$ and $0<|x-c|<\delta_2$ thus $|f(x)-L|<\epsilon/2$ and $|g(x)-M|<\epsilon/2$.  Note that $|f(x)-L+g(x)-M|\leq |f(x)-L|+|g(x)-M|<\epsilon$ so $|h(x)-(M+L)|<\epsilon$.
\end{proof}
\end{enumerate}
\begin{exercise}

4.2.7 \\
Let $g:A\rightarrow\mathbb{R}$ and assume f is a bounded function on A, in the since that there exists $M>0$ such that $f(x)<M$ for all $x\in A$.  Show that if $g(x)\rightarrow 0$ as $x\rightarrow c$ that $g(x)f(x)\rightarrow 0$.
\end{exercise}
Define $h(x)=g(x)f(x)$.  Choose $\epsilon>0$.  There must exist $\delta_0>0$ such that for all $0<|x-c|<\delta_0$, $|g(x)|<\epsilon/M$.  Choose $x$ such that $0<|x-c|<\delta_0$.  Note that $|h(x)|=|g(x)||f(x)|<(\epsilon/M)(M)=\epsilon$, thus $h(x)\rightarrow 0$ as $x\rightarrow c$.

\begin{exercise}

4.2.11 \\
Let $f(x)\leq g(x) \leq h(x)$ for all $x$ in some common domain $A$.  If as $x\rightarrow c$, $f(x)\rightarrow L$ and $h(x)\rightarrow L$ show that $g(x)\rightarrow L$ as well.
\end{exercise}
Choose a arbitrary sequence $a_n\rightarrow c$ in $A-\{c\}$.  Define $f_n=f(a_n)$, $g_n=g(a_n)$, and $h_n=h(a_n)$.  Note that $f_n\rightarrow L$ and $h_n\rightarrow L$, and that $f_n\leq g_n\leq h_n$.  By the squeeze therm on sequences $g_n\rightarrow L$.  Since $a_n$ was chosen arbitrarily we can say that $g(x)\rightarrow L$ as $x\rightarrow c$.

\begin{exercise}

4.3.1 \\
Define $g(x)=x^3$.
\end{exercise}
\begin{enumerate}[(a)]
\item
Prove that $g(x)$ is continuous at $x=0$.
\begin{proof}
Choose $\epsilon>0$.  Define $\delta=\sqrt[3]{\epsilon}$.  Choose $|x|<\delta$.  Note that $-\delta<x<\delta$ so $-\epsilon<x^3-0<\epsilon$ or $|x^3-0|<\epsilon$.  Note that $0^3=0$.
\end{proof}
\item
Prove that $g(x)$ is continuous at $c\neq 0$.
\begin{proof}
Choose $c\neq 0$.  Choose $\epsilon>0$.  Define $0<\delta=\min(\sqrt[3]{\epsilon/2},\epsilon/(9c^2),|c/2|)$.  Note that $-c/2 \leq \delta\leq c/2$.  Choose $|x-c|<\delta$.  Note that $|x^3-c^3|<\delta|x^2+xc+c^2|=\delta|x^2-2xc+c^2+3xc|\leq\delta(|x-c||x-c|+|3xc|)<\delta^3+|3xc|\delta$.  Note that $|x-c|<\delta$, $c-\delta<x<c+\delta$, $c/2<x<3c/2$, $0<3c^2/2<3xc<9c^2/2$.  Note $|x^3-c^3|<\delta^3+\delta9c^2/2\leq \epsilon/2+\epsilon/2=\epsilon$.
\end{proof}
\end{enumerate}
\begin{exercise}

4.3.3 \\
\end{exercise}
\begin{enumerate}[(a)]
\item
Prove therm 4.3.9.\\
\begin{proof}
Suppose $f:A\rightarrow \mathbb{R}$ and $g:B\rightarrow \mathbb{R}$.  Suppose $g(f(x))$ is defined for all $x\in A$.  Suppose $f$ is continuous at $c$ and $g$ is continuous at $f(c)$.  Choose $\epsilon>0$.  There must exist a $\delta_1>0$ such that for all $|y-f(c)|<\delta_1$, $|g(y)-g(f(c))|<\epsilon$.  There must exist a $\delta_2>0$ such that for all $|x-c|<\delta_2$, $|f(x)-f(c)|<\delta_1$.  Choose $|x-c|<\delta_2$.  Note that $|f(x)-f(c)|<\delta_1$, and thus $|g(f(x))-g(f(c))|<\epsilon$.
\end{proof}
\item
prove again using sequential characterization.
\begin{proof}
Suppose $f:A\rightarrow \mathbb{R}$ and $g:B\rightarrow \mathbb{R}$.  Suppose $g(f(x))$ is defined for all $x\in A$.  Suppose $f$ is continuous at $c$ and $g$ is continuous at $f(c)$.  Choose $a_n\rightarrow c$ where $a_n\in A$.  Note that $b_n=f(a_n)\in B$ and since $f(x)$ is continuous at $x=c$, $b_n\rightarrow f(c)$.  Note that $h_n=g(b_n)=g(f(a_n))$, since $g(y)$ is continuous at $y=f(c)$, $h_n\rightarrow g(f(c))$.

\end{proof}
\end{enumerate}
\begin{exercise}

4.3.5 \\
Prove that if $c$ is a isolated point of $A$ then $f:A\rightarrow \mathbb{R}$ is continuous at $c$.
\end{exercise}
\begin{proof}
Suppose $c$ is a isolated point of $A$ and $f:A\rightarrow \mathbb{R}$.  Choose $\epsilon>0$.  Since $c$ is a isolated point of $A$ there exists a $\delta$ such that $v_\delta(c) \cap A =\{c\}$.  Choose $x\in A$, $|x-c|<\delta$.  Note that there is only one $x$ with this property, thus $x=c$.  Note that $|f(x)-f(c)|=0<\epsilon$.
\end{proof}
\begin{exercise}

4.3.9 \\
If $h:\mathbb{R}\rightarrow\mathbb{R}$ and $h$ is continuous for all $\mathbb{R}$ then $\{x:h(x)=0\}$ is a closed set.
\end{exercise}
\begin{proof}
Suppose $h:\mathbb{R}\rightarrow\mathbb{R}$ and $h$ is continuous for all $\mathbb{R}$.  Suppose $\{x:h(x)=0\}$ is not a closed set.  Since $H=\{x:h(x)=0\}$ is not a closed set there must be a limit point $l$ of $H$ where $l\not\in H$.  Consider the set $a_n$ where $a_1\in v_1(l)\cap H-\{l\}$, note that $v_\epsilon(l)\cap H-\{l\}\neq \emptyset$ for any $\epsilon>0$.  And $a_n\in v_{|a_{n-1}-l|/2}(l)\cap H-\{l\}$.  Note that $-1/2^{n-1}+l\leq a_n\leq 1/2^{n-1}+l$ by construction and thus $a_n\rightarrow l$ by the squeeze therm.  Define $h_n=h(a_n)$.  Note that $a_n\in H$  thus $h_n=0$.  Since $h(x)$ is continuous and $a_n\rightarrow l$ and $h_n\rightarrow 0$ we can say that $h(l)=0$.  Therefore $l\in H$, a contradiction.
\end{proof}
\begin{exercise}

10
\end{exercise}
\begin{enumerate}[a)]
\item
Show that a continuous function on all of $\mathbb{R}$ that equals zero on the rational numbers must be the zero function.
\begin{proof}
Suppose $H$ is closed set where $\mathbb{Q}\subseteq H \subseteq \mathbb{R}$.  Choose $a\in\mathbb{R}$.  Choose $\epsilon>0$.  Note that there exist a rational $q$ such that $a<q<a+\epsilon$, by the density of the rationals.  Note that $q\in H-\{a\}$ and that $q\in v_\epsilon(a)$, therefore $a$ is a limit point of $H$ and since $H$ is closed $a\in H$ thus $\mathbb{R}\subseteq H$ and so $H=\mathbb{R}$.\\\\
Suppose $h:\mathbb{R}\rightarrow\mathbb{R}$ that equals zero on the rational numbers and $h$ is continuous for all $\mathbb{R}$.  From the previous proof $H=\{x:h(x)=0\}$ is a closed set.  Note that $H$ is closed set where $\mathbb{Q}\subseteq H \subseteq \mathbb{R}$.  Conclude $H=\mathbb{R}$, $h(x)=0$ for all $x\in\mathbb{R}$.
\end{proof}
\item
Suppose f and g are two continuous functions on the real numbers. Is it true that if $f(q)=g(q)$ for all $q \in Q$, then f and g are the same function?\\
Yes.\\
\begin{proof}
Suppose f and g are two continuous functions on the real numbers where $f(q)=g(q)$ for all $q\in\mathbb{Q}$.  Suppose $F=f(l)\neq g(l)=G$ for some $l\in\mathbb{R}$.  Define $\epsilon=|F-G|/2>0$.  There must exist a $\delta_1$ such that for all $|x-l|<\delta_1$, $|f(x)-F|<\epsilon$.  There must exist a $\delta_2$ such that for all $|x-l|<\delta_2$, $|g(x)-G|<\epsilon$.  Define $\delta=\min(\delta_1,\delta_2)$.  Note that there exists a rational $q$ such that $l-\delta<q<l+\delta$.  Note that $|g(q)-G|<\epsilon$ and that $|f(q)-F|<\epsilon$, also note that $f(q)=g(q)$.  Note that $2\epsilon=|F-G|=|g(q)-G+F-f(q)|\leq |g(q)-G|+|F-f(q)|<2\epsilon$, a contradiction we conclude the negation of our supposition, that $f(l)=g(l)$ for all $l\in\mathbb{R}$.
\end{proof}
\end{enumerate}






\newpage
\begin{exercise}

4.2.9 \\
For infinite limits we replace the arbitrarily small $\epsilon>0$ with the arbitrarily large $M>0$.
\end{exercise}\W
\begin{enumerate}[(a)]
\item
Prove $\lim_{x\rightarrow 0}1/x^2=\infty$
\begin{proof}
Choose $M>0$.  Noting that $\sqrt{M}>0$ there must exist a $\delta$ such that $1/\delta<\sqrt{M}$.  Choose $x$ such that $0<|x|<\delta$.  This means that $0<x^2<\delta^2<1/M$, so $1/x^2>M$.
\end{proof}
\item
I would define $\lim_{x\rightarrow \infty} f(x)=L$ as for any $\epsilon>0$ there exists a $M$ such that if $x>M$, $|f(x)-L|<\epsilon$.\\
Show that $\lim_{x\rightarrow \infty} 1/x=0$.
\begin{proof}
Choose $\epsilon>0$.  There must exist a $M>0$ such that $1/M<\epsilon$.  Choose $x>M$.  Note that $x>M>0$ means that $|1/x|=1/x<1/M<\epsilon$.
\end{proof}
\item
Define $\lim_{x\rightarrow \infty}f(x)=\infty$ as  for any $K>0$ there exists a $M$ such that if $x>M$, $f(x)>K$.\\
As a example $\lim_{x\rightarrow \infty}x=\infty$.
\end{enumerate}
\end{document}