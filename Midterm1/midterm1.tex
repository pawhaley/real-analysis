%%%%%%%%%%%%%%%%%%%%%%%%%%%%%%%%%%%%%%%%%%%%%%%%%%%%%%%%%%%%%%%%%%%%%%%%%%%%%%%%%%%%%%%
%%%%%%%%%%%%%%%%%%%%%%%%%%%%%%%%%%%%%%%%%%%%%%%%%%%%%%%%%%%%%%%%%%%%%%%%%%%%%%%%%%%%%%%
% 
% This top part of the document is called the 'preamble'.  Modify it with caution!
%
% The real document starts below where it says 'The main document starts here'.

\documentclass[12pt]{article}

\usepackage{amssymb,amsmath,amsthm}
\usepackage[top=1in, bottom=1in, left=1.25in, right=1.25in]{geometry}
\usepackage{fancyhdr}
\usepackage{enumerate}
\usepackage{color}

% Comment the following line to use TeX's default font of Computer Modern.
\usepackage{times,txfonts}

\newtheoremstyle{homework}% name of the style to be used
  {18pt}% measure of space to leave above the theorem. E.g.: 3pt
  {12pt}% measure of space to leave below the theorem. E.g.: 3pt
  {}% name of font to use in the body of the theorem
  {}% measure of space to indent
  {\bfseries}% name of head font
  {:}% punctuation between head and body
  {2ex}% space after theorem head; " " = normal interword space
  {}% Manually specify head
\theoremstyle{homework} 

% Set up an Exercise environment and a Solution label.
\newtheorem*{exercisecore}{Exercise \@currentlabel}
\newenvironment{exercise}[1]
{\def\@currentlabel{#1}\exercisecore}
{\endexercisecore}

\newcommand\W{{\color{red}\textbf{(W) (Hand this one in to David.)}}}
\newcommand\tome{{\color{red}\textbf{(Hand this one in to David.)}}}

\newcommand{\localhead}[1]{\par\smallskip\noindent\textbf{#1}\nobreak\\}%
\newcommand\solution{\localhead{Solution:}}

%%%%%%%%%%%%%%%%%%%%%%%%%%%%%%%%%%%%%%%%%%%%%%%%%%%%%%%%%%%%%%%%%%%%%%%%
%
% Stuff for getting the name/document date/title across the header
\makeatletter
\RequirePackage{fancyhdr}
\pagestyle{fancy}
\fancyfoot[C]{\ifnum \value{page} > 1\relax\thepage\fi}
\fancyhead[L]{\ifx\@doclabel\@empty\else\@doclabel\fi}
\fancyhead[C]{\ifx\@docdate\@empty\else\@docdate\fi}
\fancyhead[R]{\ifx\@docauthor\@empty\else\@docauthor\fi}
\headheight 15pt

\def\doclabel#1{\gdef\@doclabel{#1}}
\doclabel{Use {\tt\textbackslash doclabel\{MY LABEL\}}.}
\def\docdate#1{\gdef\@docdate{#1}}
\docdate{Use {\tt\textbackslash docdate\{MY DATE\}}.}
\def\docauthor#1{\gdef\@docauthor{#1}}
\docauthor{Use {\tt\textbackslash docauthor\{MY NAME\}}.}
\makeatother

% Shortcuts for blackboard bold number sets (reals, integers, etc.)
\newcommand{\Reals}{\ensuremath{\mathbb R}}
\newcommand{\Nats}{\ensuremath{\mathbb N}}
\newcommand{\Ints}{\ensuremath{\mathbb Z}}
\newcommand{\Rats}{\ensuremath{\mathbb Q}}
\newcommand{\Cplx}{\ensuremath{\mathbb C}}
%% Some equivalents that some people may prefer.
\let\RR\Reals
\let\NN\Nats
\let\II\Ints
\let\CC\Cplx

%%%%%%%%%%%%%%%%%%%%%%%%%%%%%%%%%%%%%%%%%%%%%%%%%%%%%%%%%%%%%%%%%%%%%%%%%%%%%%%%%%%%%%%
%%%%%%%%%%%%%%%%%%%%%%%%%%%%%%%%%%%%%%%%%%%%%%%%%%%%%%%%%%%%%%%%%%%%%%%%%%%%%%%%%%%%%%%
% 
% The main document start here.

% The following commands set up the material that appears in the header.
\doclabel{Math 401: Midterm}
\docauthor{Parker Whaley}
\docdate{Due October 19, 2016}

\begin{document}
Note that I am operating under the convention that $N,n,m,i,j$ are natural numbers unless otherwise specified.
\begin{exercise}

Let $A$ and $B$ be nonempty sets that are bounded above. Suppose $\sup A < \sup B$. Prove that there is an element of $B$ that is an upper bound for $A$.
\end{exercise}
\begin{proof}
Suppose $A$ and $B$ are nonempty sets that are bounded above. Furthur suppose $\sup A < \sup B$. Define $a=\sup A$ and $b=\sup B$.  Note that $a$ is less than the suppremum of $B$ thus $a$ is not a upper bound on $B$.  Since $a$ is not a upper bound on $B$ there must exist at least one element of $B$ grater than $a$, take one of these elements lets call it $k$, $k\in B$, $a<k$.  Choose a arbitrary element $c\in A$.  Since $a=\sup A$ we know that $a$ is a upper bound on $A$ therfore $c\leq a < k$.  Since we chose a arbitrary element from $A$ and showed that it is less than $k$ we can say that all elements in $A$ are less then $k$ thus $k$ is a upper bound on $A$.
\end{proof}

\begin{exercise}

In class we proved that $\mathbb{N}^2$ is countably infinite. Use this fact and a proof by induction to show that $\mathbb{N}^n$ is countably infinite for every $n \in \mathbb{N}$.
\end{exercise}
\begin{proof}
We want to show that for every $n \in \mathbb{N}$, $\mathbb{N}^n$ is countably infinite.  I will procede with a proof by induction.\\
Base case $n=1$.  There is a bijective map, the identity map, mapping $\mathbb{N}^1\rightarrow\mathbb{N}$.  So the statement holds in the $n=1$ case.\\
Suppose $\mathbb{N}^m$ is countably infinite for all $m\leq n$ where $n\geq 1$.  There must exist a bijective map from $\mathbb{N}^n\rightarrow \mathbb{N}$, the definition of countably infinate.  Note that $\mathbb{N}^{n+1}$ can trivially be bijectively mapped to $\mathbb{N}^n\times \mathbb{N}$, by mapping the first term to $\mathbb{N}$ and the next terms to $\mathbb{N}^n$.  Note that there exists a bijective map from $\mathbb{N}^2\rightarrow \mathbb{N}$ since $\mathbb{N}^2$ is countably infinate.  Note that we can bijectively map $\mathbb{N}^{n+1}\rightarrow \mathbb{N}^n\times \mathbb{N}\rightarrow \mathbb{N}\times \mathbb{N}\rightarrow \mathbb{N}^2\rightarrow \mathbb{N}$.  Thus $\mathbb{N}^{n+1}$ is countably infinate.\\
By induction we can conclude that for every $n \in \mathbb{N}$, $\mathbb{N}^n$ is countably infinite.
\end{proof}
\begin{exercise}

Compute $$\lim_{n\rightarrow\infty}\frac{3^n}{n!}$$.
\end{exercise}
Define $a_n=0$,$b_n=\frac{3^n}{n!}$,$c_n=\frac{3^n}{4^{n-4}}$.  Note that for $1\leq n\leq 4$, $b_n=\frac{3^n}{n!}\leq 3^n\leq \frac{3^n}{4^{n-4}}=c_n$.  Note that for $n=4$, $b_n\leq c_n$.  Suppose for some $n\geq 4$, $b_n\leq c_n$.  Note that $b_{n+1}=\frac{3^{n+1}}{(n+1)!}=\frac{3^{n}}{n!}\frac{3}{n+1}=b_n\frac{3}{n+1}\leq c_n\frac{3}{n+1}\leq c_n\frac{3}{4}=\frac{3^n}{4^{n-4}}\frac{3}{4}=\frac{3^{n+1}}{4^{(n+1)-4}}=c_{n+1}$.  By induction I conclude that for all $n\geq 4$, $b_n\leq c_n$.  So for all $n\in \mathbb{N}$, $b_n\leq c_n$.  Also note that $b_n$ is always positive and thus $a_n\leq b_n$.  Note that $a_n\rightarrow 0$.  Note that $c_n$ is bounded bellow by $0$.  Also note that $c_{n+1}=\frac{3}{4}c_n\leq c_n$, so $c_n$ is monotone decreasing and bounded below, by the MCT it must converge.  Define $l$ as the limit of $c_n$.  Note that $c_{n+1}\rightarrow l$ also note that $c_{n+1}=\frac{3}{4}c_n\rightarrow \frac{3}{4}l$.  So $l=\frac{3}{4}l$ therfore $l=0$.  By the squeze teorem $\frac{3^n}{n!}\rightarrow 0$.

\begin{exercise}

Suppose $F$ is a collection of open intervals such that if $I, J \in F$ and $I \neq J$, then $I \cap J = \emptyset$.
Prove that $F$ is countable.\\
\begin{proof}
Suppose $F$ is a collection of open intervals such that if $I, J \in F$ and $I \neq J$, then $I \cap J = \emptyset$.\\
If $F$ is finite then it is at most countable.\\
Suppose $F$ is non-finite.  Select one element of $F$, let's call it $d$ (for default).  Consider the mapping $f:\mathbb{Q}\rightarrow G$ where $G=P(F)$, the power set of $F$, so that $G$ is the set of all subsets of $F$.
$$f(q)=\begin{cases}
\{f\in F: q\in f\} & \{f\in F: q\in f\}\neq \emptyset\\
\{d\} & $otherwise$
\end{cases}$$
Suppose there existed a $q$ such that $f(q)$ did not have cardinality $1$.  Note that $f(q)\neq \emptyset$, since we map anything that would have maped to the emptyset to the set containing the default set.  Thus $f(q)$ must have at least two elements $I, J\in f(q)$.  Also note that $f(q)\neq \{d\}$ since that has cardinality of one, $f(q)=\{f\in F: q\in f\}$.  Note that by our above construction $I, J \in F$ and $I\neq J$ and $q\in I$, $q\in J$, since $f(q)\neq \{d\}$.  Thus $q\in I\cap J$ so $I\cap J\neq \emptyset$.  This is a contradiction since our initial upposition tells us $I \cap J = \emptyset$, we conclude the negation of our supposition, that $f(q)$ has one element for all $q\in \mathbb{Q}$.\\
We can now construct a function $g:\mathbb{Q}\rightarrow F$, where $g(q)$ is the one element in $f(q)$, noting that $f(q)\in G$ means that $f(q)\subseteq F$ and thus the one element in $f(q)$ is a element of $F$.\\
Choose $I\in F$.  Note that by the density of the rationals there is a rational in the open interval $I$, select one of these elements and call it $q$, $q\in I$ and $q\in \mathbb{Q}$.  Note that $g(q)$ is the element in $f(q)$, and $I\in \{f\in F: q\in f\}\neq \emptyset$ thus $f(q)=\{f\in F: q\in f\}$ and $I\in f(q)$ thus $g(q)=I$.  Since we chose a arbitrary element in $F$ and found a $q\in\mathbb{Q}$ that maps to it via $f$ we can say $f$ is onto.  We know that there is a onto map $h:\mathbb{N}\rightarrow\mathbb{Q}$ since $\mathbb{N}$ and $\mathbb{Q}$ have the same cardinality.  Consider the map $m:\mathbb{N}\rightarrow F$ where $m(n)=g(h(n))$.  Note that $m$ is onto.  Since there is a onto map from $\mathbb{N}\rightarrow F$ we know that $F$ is at most countably infinate.
\end{proof}
\end{exercise}
\end{document}