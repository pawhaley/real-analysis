%%%%%%%%%%%%%%%%%%%%%%%%%%%%%%%%%%%%%%%%%%%%%%%%%%%%%%%%%%%%%%%%%%%%%%%%%%%%%%%%%%%%%%%
%%%%%%%%%%%%%%%%%%%%%%%%%%%%%%%%%%%%%%%%%%%%%%%%%%%%%%%%%%%%%%%%%%%%%%%%%%%%%%%%%%%%%%%
% 
% This top part of the document is called the 'preamble'.  Modify it with caution!
%
% The real document starts below where it says 'The main document starts here'.

\documentclass[12pt]{article}

\usepackage{amssymb,amsmath,amsthm}
\usepackage[top=1in, bottom=1in, left=1.25in, right=1.25in]{geometry}
\usepackage{fancyhdr}
\usepackage{enumerate}
\usepackage{color}

% Comment the following line to use TeX's default font of Computer Modern.
\usepackage{times,txfonts}

\newtheoremstyle{homework}% name of the style to be used
  {18pt}% measure of space to leave above the theorem. E.g.: 3pt
  {12pt}% measure of space to leave below the theorem. E.g.: 3pt
  {}% name of font to use in the body of the theorem
  {}% measure of space to indent
  {\bfseries}% name of head font
  {:}% punctuation between head and body
  {2ex}% space after theorem head; " " = normal interword space
  {}% Manually specify head
\theoremstyle{homework} 

% Set up an Exercise environment and a Solution label.
\newtheorem*{exercisecore}{Exercise \@currentlabel}
\newenvironment{exercise}[1]
{\def\@currentlabel{#1}\exercisecore}
{\endexercisecore}

\newcommand\W{{\color{red}\textbf{(W) (Hand this one in to David.)}}}
\newcommand\tome{{\color{red}\textbf{(Hand this one in to David.)}}}

\newcommand{\localhead}[1]{\par\smallskip\noindent\textbf{#1}\nobreak\\}%
\newcommand\solution{\localhead{Solution:}}

%%%%%%%%%%%%%%%%%%%%%%%%%%%%%%%%%%%%%%%%%%%%%%%%%%%%%%%%%%%%%%%%%%%%%%%%
%
% Stuff for getting the name/document date/title across the header
\makeatletter
\RequirePackage{fancyhdr}
\pagestyle{fancy}
\fancyfoot[C]{\ifnum \value{page} > 1\relax\thepage\fi}
\fancyhead[L]{\ifx\@doclabel\@empty\else\@doclabel\fi}
\fancyhead[C]{\ifx\@docdate\@empty\else\@docdate\fi}
\fancyhead[R]{\ifx\@docauthor\@empty\else\@docauthor\fi}
\headheight 15pt

\def\doclabel#1{\gdef\@doclabel{#1}}
\doclabel{Use {\tt\textbackslash doclabel\{MY LABEL\}}.}
\def\docdate#1{\gdef\@docdate{#1}}
\docdate{Use {\tt\textbackslash docdate\{MY DATE\}}.}
\def\docauthor#1{\gdef\@docauthor{#1}}
\docauthor{Use {\tt\textbackslash docauthor\{MY NAME\}}.}
\makeatother

% Shortcuts for blackboard bold number sets (reals, integers, etc.)
\newcommand{\Reals}{\ensuremath{\mathbb R}}
\newcommand{\Nats}{\ensuremath{\mathbb N}}
\newcommand{\Ints}{\ensuremath{\mathbb Z}}
\newcommand{\Rats}{\ensuremath{\mathbb Q}}
\newcommand{\Cplx}{\ensuremath{\mathbb C}}
%% Some equivalents that some people may prefer.
\let\RR\Reals
\let\NN\Nats
\let\II\Ints
\let\CC\Cplx

%%%%%%%%%%%%%%%%%%%%%%%%%%%%%%%%%%%%%%%%%%%%%%%%%%%%%%%%%%%%%%%%%%%%%%%%%%%%%%%%%%%%%%%
%%%%%%%%%%%%%%%%%%%%%%%%%%%%%%%%%%%%%%%%%%%%%%%%%%%%%%%%%%%%%%%%%%%%%%%%%%%%%%%%%%%%%%%
% 
% The main document start here.

% The following commands set up the material that appears in the header.
\doclabel{Math 401: Homework 2}
\docauthor{Parker Whaley}
\docdate{September 12, 2016}

\begin{document}

\begin{exercise}{1.4.1}
Recall that $\mathbb{I}$ stands for the set of irrational numbers.
\begin{enumerate}[(a)]
\item Show that if $a,b\in\Rats$ then $ab$ and $a+b\in\Rats$ as well.
\item Show that if $a\in\Rats$ and $t\in\mathbb{I}$ then $a+t\in\mathbb{I}$ and if $a\neq 0$ then $at\in\mathbb{I}$ as well.
\item Part (a) says that the rational numbers are closed under multiplication
and addition.  What can be said about $st$ and $s+t$ when $s,t\in\mathbb{I}$?
\end{enumerate}
\end{exercise}
\begin{enumerate}[(a)]
\item \begin{proof}
Select two arbitrary elements from the rational numbers, since they are rational we can represent them as $i/j$ and $m/n$ where $i,j,m,n\in \mathbb{Z}$ and $j\neq 0$ and $n\neq 0$.\\\\
Note that $i/j*m/n=\frac{im}{jn}$, from the definition of multiplication of rational numbers.  Since the multiple of any two non-zero numbers is non-zero and since the multiple of any two integers is a integer so $jn\in \mathbb{Z}-\{0\}$ and $im\in \mathbb{Z}$ therfore $\frac{im}{jn}\in \mathbb{Q}$.\\\\
Note that $i/j+m/n=\frac{in+mj}{jn}$, from the definition of addition of rational numbers.  Since the multiple of any two non-zero numbers is non-zero and since the multiple of any two integers is a integer so $jn\in \mathbb{Z}-\{0\}$ and $in,mj\in \mathbb{Z}$ and so also $in+mj \in \mathbb{Z}$ therfore $\frac{in+mj}{jn}\in \mathbb{Q}$.
\end{proof}

\item \begin{proof}
Proof by contradiction.\\
Suppose that there exists $a\in\Rats$ and $t\in\mathbb{I}$ where $a+t=b\notin\mathbb{I}$.  Since the reals are closed under addition we can say $b\in\mathbb{R}$.  Note that $b\in\mathbb{R} -\mathbb{I}=\mathbb{Q}$.  We can do a little math and see $a+t=b$ means $t=b+(-a)$.  Since the addative inverse of a rational is a rational and since the sum of two rationals is rational we conclude $t\in \mathbb{Q}$.  A contradiction has been reached, the rationals and irrationals are, by deffinition, mutualy exclusive and so $t$ cannot be a element of both.\\\\
Proof by contradiction.\\
Suppose that there exists $a\in\Rats-\{0\}$ and $t\in\mathbb{I}$ where $at=b\notin\mathbb{I}$.  Since the reals are closed under multiplication, and since the multiple of any two non zero numbers is itself non zero, we can say $b\in\mathbb{R}-\{0\}$.  Note that $b\in\mathbb{R}-\{0\} -\mathbb{I}=\mathbb{Q}-\{0\}$.  We can do a little math and see $at=b$ means $t=b(a^{-1})$.  Since the multiplicative inverse of a non zero rational is a rational (informaly $(\frac{i}{m})(\frac{m}{i})=1$) and since the multiple of two rationals is rational we conclude $t\in \mathbb{Q}$.  A contradiction has been reached, the rationals and irrationals are, by deffinition, mutualy exclusive and so $t$ cannot be a element of both.
\end{proof}

\item 
All we can conclude is that $st\in \mathbb{R}-\{0\}$ and that $s+t\in \mathbb{R}$.  As a example that the irrationals are not closed with respect to multiplication or addition note that $\sqrt{2}\sqrt{2}=2$ and that $\pi+(-\pi)=0$.
\end{enumerate}

\begin{exercise}{1.4.2}
Let $A\subseteq \Reals$ be nonempty and bounded above. Let $s\in\Reals$ have
the property that for all $n\in\Nats$, $s+(1/n)$ is an upper bound for $A$
but $s-(1/n)$ is not an upper bound for $A$.  Show that $s=\sup A$.
\end{exercise}
\begin{proof}
Take some $a\in\mathbb{R}$ where $s<a$.  Note $a-s\in \mathbb{R}^+$ and so by the proof we previously compleated in class there is a natural number $n$ with the property $\frac{1}{n}<a-s$.  So $s+\frac{1}{n}<a$ wich means $a$ is bigger than a upper bound on $A$ and so is itself not in $A$.  Since we chose a arbitrary real grater than $s$ and showed it is not in $A$ we can conclude no real biger than $s$ is in $A$, in other words all elements of $A$ are less than or equal to $s$.  By definition $s$ is a upper bound on $A$.\\
Next take some $b\in\mathbb{R}$ where $b<a$.  Note $s-b\in \mathbb{R}^+$ and so by the proof we previously compleated in class there is a natural number $n$ with the property $\frac{1}{n}<s-b$.  So $b<s-\frac{1}{n}$ wich means $b$ is smaller than a number that is not a upper bound on $A$.  Noting that if $b$ were upper bound on $A$ we would be forced to conclude $s-\frac{1}{n}$ is a upper bound on $A$, wich we know to be falce, we are forced to conclude $b$ is not a upper bound on A.  and since $b$ was chosen arbitraraly we can conclude all upper bounds on $A$ are grater than or equal to $s$.\\
Noting that $s$ has both properties of a suppremum we conclude $s=\sup A$.
\end{proof}

\begin{exercise}{1.4.3} Show that $\cap_{n=1}^\infty (0,1/n)=\emptyset$.
\end{exercise}
\begin{proof}
Proof by contradiction.\\
Suppose $a\in \cap_{n=1}^\infty (0,1/n)$.  Note that $a\in (0,1) \subseteq \mathbb{R}^+$ so $a\in\mathbb{R}^+$.  By the proof done in class there exists a natural i with the property $1/i<a$.  Since a is in the intersect of all of the $(0,1/n)$ sets $a\in (0,1/i)$ so $a<1/i$ a contradiction.  We now conclude the negation of our supposition namely there is no $a$ with the property $a\in \cap_{n=1}^\infty (0,1/n)$ or a logicaly equivelent statement $\cap_{n=1}^\infty (0,1/n)=\emptyset$.
\end{proof}

\begin{exercise}{1.4.4} \W 

Let $a<b$ be real numbers and let $T=[a,b]\cap\Rats$.
Show that $\sup T=b$.
\end{exercise}
\begin{proof}
First we note that $b$ is a upper bound on $T$, since $T\subseteq [a,b]$ and eavery element of $[a,b]$ is less than or equal to b.\\
Take some real number $c<b$.  Note that $b-c\in\mathbb{R}^+$ and thus there exists a $n\in\mathbb{N}$ where $1/n<b-c$.  Note $cn\in\mathbb{R}$ and so there are natural numbers larger than $cn$, take the smallest of these lets call it m so $m-1<cn<m$ .
\end{proof}

\begin{exercise}{1.4.5} Use Exercise 1.4.1 to provide a proof of Corollary 1.4.4 by considering real numbers $a-\sqrt{2}$ and $b-\sqrt{2}$.
\end{exercise}
\begin{proof}
\end{proof}



\begin{exercise}{Supplemental 1} Show that the sets $[0,1)$ and $(0,1)$ have the same cardinality.
\end{exercise}

\end{document}